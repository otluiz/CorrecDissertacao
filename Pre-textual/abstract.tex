\vspace*{12pt}
% Abstract

% Selecionar a linguagem acerta os padrões de hifenação diferentes entre inglês e português.
%\selectlanguage{english}

\noindent{Federal highways that cross the Metropolitan
	Regions of some cities are constantly congested, not only by the
	number of vehicles, but also because they are subject to
	stoppages of the most diverse natures, such as workers' protests,
	accidents, holes, natural weather and other types of problems. In
	extreme situations they could even bring to a halt the production
	of factories in the surroundings of such roads. We propose a
	classification model of behavioral patterns for the federal
	highways that cross the metropolitan area of Pernambuco, which
	is a state in the Northeastern region of Brazil. The proposed
	model allows some events anticipation, especially those that may
	cause constraints, retention, or reduction of traffic flow. The data
	source of this research is from the database of the Federal
	Highway Police of Pernambuco (PRF) since 2007 until 2016. We
	have considered vehicles, track layout and road sections related
	to accidents, among others. Based on the information obtained, a
	Data Mining was performed using the CRISP-DM methodology
	to find behavioral patterns on highways and in their
	surroundings. Machine learning algorithms were used for
	classification and regression, being prioritized, Decision Trees
	and Neural Networks. The values of the area under the ROC
	(AUC) curve obtained were above 0.7 which reflects a good
	degree of reliability. The proposed prediction model means an
	advance in terms of mobility and cargo transport management,
	since it allows anticipating events and behaviors, favoring the
	choice of alternative routes and increasing the time for choice for
	certain routes 
}

\par
\vspace{1em}
\noindent\textbf{Keywords: Data Mining, Data Bases, Social Network, Logistic, Routing}