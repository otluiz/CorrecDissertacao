\vspace*{12pt}
% Abstract

% Selecionar a linguagem acerta os padrões de hifenação diferentes entre inglês e português.
%\selectlanguage{english}

\noindent{The federal highways that cross at Metropolitan Region of some cities are 	
	constantly congested, not only because the number of vehicles, but due to downtime, 
	such as worker protests, accidents, natural events and other types of congestion 	
	factors. In extreme situations these problems could paralyze even the production 	
	of factories in their surroundings, causing great losses. Thus, this research aimed 	
	propose and test concepts for a self-adaptive platform that contemplates a 	
	predictive model of behavior of the federal highways that cross the state of 	
	Pernambuco (Brazil), so that it is possible to anticipate events that may occur in 	
	certain stretches of highway that may cause embarrassment, such as 	
	traffic reduction and downtime. The primary data source of this research comes 	
	from the Federal Highway Police of Pernambuco (PRF/PE) database from 2007 to 2015	
	onwards, having considered vehicles, track layout and road sections related to 	
	accidents. Data from the social network Twitter, of the last years, both from 	
	the PFR, and from people who mentioned events in BRs (accidents, stoppages, 	
	etc.).
	Based on the information obtained, a Data Mining was performed using the 
	CRISP-DM methodology to find behavioral patterns on the roads and in its 
	surroundings. The technologies used for Mining were: Decision Trees, Naïve Bayes and Neural Networks.
	The values of the area under the ROC curve (AUC) obtained were above 0.8 which reflects a good degree of reliability. 
	With Twitter data, all the tweets for each keyword were collected up to the limits 
	of the application. The technologies used were Naïve Bayes and TF-IDF and,
	to display geolocation, we used Google Maps georeferencing software.
	Compared to usual navigation approaches, the proposed prediction model 
	represents a breakthrough in terms of mobility and management of tranportation and vehicle , since it makes it possible to anticipate events and behaviors, in order to favor the selection of alternative routes.
}

\par
\vspace{1em}
\noindent\textbf{Keywords: Prediction Model, Data Mining, CRISP-DM, Road Traffic Control}