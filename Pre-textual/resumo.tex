\vspace*{12pt}

\lettrine {O} transporte de cargas, que atravessa as regiões metropolitanas das grandes cidades brasileiras é realizado principalmente pelas rodovias federais. 
Essas rodovias estão constantemente congestionadas e têm recebido aumento expressivo de novos veículos a cada ano. 

No entorno desses trechos urbanos têm crescido desordenadamente comunidades que demandam por políticas sociais que atendam às 
suas necessidades. Para reivindicar dos entes públicos essas comunidades bloqueiam as rodovias, aumentando a pressão sobre os congestionamentos.
Em alguns trechos o traçado das rodovias está próximo a morros e florestas ficando suscetíveis às intempéries climáticas.
Essas variáveis impõem constantes paralisações às rodovias, representando atrasos nas entregas, custos adicionais às empresas e prejuízos à competitividade nacional. 

Propor novas soluções, que venham minimizar esses constrangimentos é condição 'sine qua non', para mitigar o que vem sendo chamado de ``Custo Brasil''. 
Um modelo preditivo de sugestão de roteamento de cargas que antecipe eventos futuros utilizando informações vindas de base de dados históricas da Polícia Rodoviária Federal, 
das redes sociais e das condições socio-ambientais que exiba os resultados em forma mapas eletrônicos configuráveis e 
auto-adaptáveis irá contribuir na tomada de decisão por um gestor que poderá escolher com segurança enviar uma 
frota de veículos de cargas por determinadas rodovias antecipando prováveis eventos, que possam interferir no fluxo normal das 
rodovias brasileiras, propondo assim uma alternativa ao traçado de rotas determinísticas e incorporando novas 
opções de rotas inspiradas na predição de eventos futuros.

Assim, esse mestrado tem por objetivo desenvolver uma plataforma auto-adaptável de suporte a decisão para a problemática 
crescente da Logística de Cargas.

\par
\vspace{2em}
\noindent\textbf{Palavras-chave: Mineração de dados, Bases de dados históricas, Redes sociais, Logística de transportes, Roteamento}