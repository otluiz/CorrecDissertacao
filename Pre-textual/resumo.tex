\vspace*{12pt}

\lettrine {O}transporte de cargas que atravessa as regiões metropolitanas das grandes cidades brasileiras é realizado principalmente pelas rodovias federais. 
Essas rodovias frequentemente se encontram congestionadas em determinados dias e/ou horários.
Além do mais, tem sido contabilizado um aumento expressivo de veículos que por elas trafegam, a cada ano. No entorno de tais rodovias, particularmente em perímetros urbanos, comunidades realizam bloqueios para protestar contra acidentes, atropelamentos ou ainda paralisações de cunho político, como greves, etc.  A proximidade das rodovias de trechos com morros, florestas, rios, contribuem para que questões ligadas às intempéries da natureza, como, por exemplo, deslizamentos, promovam bloqueio das estradas. Essas variáveis impõem constantes paralisações às rodovias, representando atrasos na entrega, custos adicionais às empresas e prejuízos de várias ordens. 
Partindo dessa preocupação, esse estudo teve por objetivo propor e testar conceitos para uma plataforma auto-adaptável que 
contemple um modelo preditivo de comportamento das rodovias federais que atravessam o estado de Pernambuco, de modo que seja 
possível antecipar eventos que poderão ocorrer em determinados trechos de rodovia, que possam causar constrangimentos, 
como retenção, redução de tráfego (gargalos) e paralisação.

Para a proposição desse modelo, foram coletadas informações, a partir de 2007, na base de dados da Polícia Rodoviária Federal de Pernambuco, Polícia Rodoviária Estadual, IBGE e DataSus, INPE, além das redes sociais, como Twitter, e informações do GoogleMaps. 
Com base nas informações obtidas, foi realizada uma Mineração de Dados (FAYYAD; PIATETSKY-SHAPIRO e SMYTH, 1996; HAN e KAMBER, 2006), utilizando a metodologia CRISP-DM (CHAPMAN; KERBER; WIRTH et al, 2000) para encontrar padrões comportamentais nas rodovias e em seu entorno. As tecnologias empregadas para a Mineração foram: Redes Neurais (HEATON, 2008), Árvores de Decisão (SRIVASTAVA; KATIYAR e SINGH, 2015) e Regressão Logística (HILBE, 2009). 
Para dados atualizados foram coletadas informações do Twitter, e para exibir a localização, utilizamos o Google Maps.
O modelo de predição proposto significa um avanço em termos de mobilidade e gestão do transporte de cargas, uma vez que possibilita antecipar eventos e comportamentos, favorecendo a escolha de rotas alternativas e ampliando o espaço temporal de escolha para determinadas rotas.

\par
\vspace{2em}
\noindent\textbf{Palavras-chave: Modelo de Predição, Mineração de dados, CRISP-DM, Controle de tráfego rodoviário.}