\vspace*{12pt}

\lettrine {As} rodovias federais que atravessam a Região
Metropolitana e cidades do interior estão constantemente
congestionadas, não apenas pela quantidade de veículos,
mas por serem alvo de paralisações das mais diversas
matizes, como protestos de trabalhadores, acidentes,
danos na via, intempéries naturais e outros fatores de
congestionamento. Em situações extremas esses problemas
poderiam paralisar até a produção das fábricas no seu
entorno, causando grandes prejuízos. Para dirimir alguns
destes problemas, essa pesquisa teve por objetivo propor e testar conceitos 
para uma plataforma autoadaptável que contemple um modelo preditivo de comportamento das rodovias federais que
atravessam o estado de Pernambuco na região Nordeste do
Brasil, de modo que seja possível, antecipar eventos que
possam vir causar constrangimentos, como retenção, redução do fluxo de tráfego (gargálos) e paralisação. A fonte primária de dados
dessa pesquisa provém da base de dados da Polícia 
Rodoviária Federal de Pernambuco (PRF/PE) entre 2007 e 2015 tendo considerado veículos, traçado da via e trechos da
rodovia relacionados a acidentes. Foram também utilizados dados da rede social Twitter dos últimos anos, tanto da PRF quanto de 
pessoas que fizeram menção a acontecimentos nas BR's (acidentes, paralisações, etc) no Estado de Pernambuco. 
Com base nas informações obtidas, foi realizada uma Mineração de Dados utilizando a metodologia CRISP-DM, além de Mineração de Textos para encontrar padrões comportamentais nas rodovias e em seu
entorno. As tecnologias empregados para a mineração foram: Árvores de Decisão, Naïve Bayes e Redes Neurais. 
Os valores da área sob a curva ROC (AUC) obtidos foram
acima de 0.8 o que representa um bom grau de confiabilidade. Com os dados do Twitter foram coletados todos os tweets referentes a cada palavra chave, até o limite imposto pelo aplicativo. As tecnologias utilizadas foram Naïve Bayes, TF-IDF e, para exibir a geolocalização, utilizamos o software de georreferenciamento Google Maps.
Em comparação com abordagens usuais de navegação, o
modelo de predição proposto representa um avanço em
termos de mobilidade e gestão do transporte, tráfego em rodovias, uma vez que possibilita antecipar eventos e
comportamentos, favorecendo a escolha de rotas alternativas e ampliando o espaço temporal de escolha para determinadas rotas. 


\par
\vspace{2em}
\noindent\textbf{Palavras-chave: Modelo de Predição, Mineração de dados, CRISP-DM, Controle de tráfego rodoviário.}