\vspace*{12pt}

\lettrine {As} rodovias federais que atravessam a Região
Metropolitana de algumas cidades estão constantemente
congestionadas, não apenas pela quantidade de veículos,
mas por serem alvo de paralisações das mais diversas
matizes, como protestos de trabalhadores, acidentes,
buracos, intempéries naturais e outros tipos de fatores de
congestionamento. Em situações extremas esses problemas
poderiam paralisar até a produção das fábricas no seu
entorno, causando grandes prejuízos. Para dirimir alguns
destes problemas propomos um modelo de classificação
para o comportamento das rodovias federais que
atravessam o estado de Pernambuco na região Nordeste do
Brasil, de modo que seja possível antecipar eventos que
possam vir causar constrangimentos de tráfego, como
retenção, redução de fluxo de tráfego. A fonte de dados
dessa pesquisa provém da base de dados da Polícia
Rodoviária Federal de Pernambuco (PRF) a partir de 2007
tendo considerado veículos, traçado da via e trechos da
rodovia relacionados a acidentes, além das redes sociais, como Twitter, e informações do GoogleMaps. Com base
nas informações obtidas, foi realizada uma Mineração de
Dados utilizando a metodologia CRISP-DM para
encontrar padrões comportamentais nas rodovias e em seu
entorno. Foram empregados algoritmos de aprendizagem
de máquina para classificação e regressão, sendo
priorizadas, Árvores de Decisão e Redes Neurais. Os
valores da área sob a curva ROC (AUC) obtidos foram
acima de 0.7 que reflete um bom grau de confiabilidade.
Em comparação com abordagens usuais de navegação, o
modelo de predição proposto representa um avanço em
termos de mobilidade e gestão do transporte de cargas,
uma vez que possibilita antecipar eventos e
comportamentos. Assim possibilitando que sejam
favorecidas rotas alternativas e ampliando o espaço
temporal de escolha para determinadas rotas. 
Partindo dessa preocupação, esse estudo teve por objetivo propor e testar conceitos para uma plataforma auto-adaptável que 
contemple um modelo preditivo de comportamento das rodovias federais que atravessam o estado de Pernambuco, de modo que seja 
possível antecipar eventos que poderão ocorrer em determinados trechos de rodovia, que possam causar constrangimentos, 
como retenção, redução de tráfego (gargalos) e paralisação.

Com base nas informações obtidas da PRF, foi realizada uma Mineração de Dados, utilizando a metodologia CRISP-DM para encontrar padrões comportamentais nas rodovias e em seu entorno. As tecnologias empregadas para a Mineração foram: Árvores de Decisão, Naïve Bayes e  Redes Neurais. 
Com os dados do Twitter foram coletadas todos os tweets referentes à cada palavra-chave até o limite imposto pelo aplicativo, as tecnologias utilizadas foram Naïve Bayes e TF-IDF e, para exibir a geolocalização, utilizamos o software de georreferenciamento QGIS.
O modelo de predição proposto significa um avanço em termos de mobilidade e gestão do transporte, uma vez que possibilita antecipar eventos e comportamentos, favorecendo a escolha de rotas alternativas e ampliando o espaço temporal de escolha para determinadas rotas.



\par
\vspace{2em}
\noindent\textbf{Palavras-chave: Modelo de Predição, Mineração de dados, CRISP-DM, Controle de tráfego rodoviário.}