Ao meu orientador Prof. Dr. Fernando Buarque, sábio e altivo, que sempre soube guiar-me pelos caminhos ``não lineares''  da pesquisa.\\

À minha mãe, referência de dedicação e perseverança. Ensinou-me quase tudo que sei, principalmente o gosto pela leitura.\\

Aos meus filhos Luiz Fellipe e Rafael Luiz, experiência enriquecedora, motivação para fazer melhor e razão para seguir sempre em frente.\\

À minha amada ``Dulcinéa'' (Anna Paula), referência de amor e dedicação, interlocutora perspicaz, sempre pronta a ouvir e dialogar. Teve muita paciência com seu cavaleiro errante ``Dom Quixote''.\\

A todos da Polícia Rodoviária Federal, pelo dados cedidos, em especial ao agente Deivierson,  sempre pronto a esclarecer minhas dúvidas.\\

A todos os professores da UPE, em especial à coordenadora Prof. Dra. Maria de Lourdes, que transformaram esta universidade em referência nacional e o PPGES em referência internacional.\\

A todos os colegas de mestrado que se transformaram em melhores amigos, em especial: ``Mega'', ``Rodrigão'', ``Felipe San'', Dupleix, ``Pastor Charles'', ``Fuzzuboy'', ``Pedro Malandro'' e tantos outros que tornaram o ambiente do PPGES alegre, saudável e fecundo em ideias.\\

A Júlia, profissional dedicada e divertida, que quando não falava muito era porque algo estava errado.\\

Aos colegas da disciplina de Mineração de Dados na UFPE, em especial Orlando e Bruno, que se tornaram grandes amigos e interlocutores para todas as horas.

