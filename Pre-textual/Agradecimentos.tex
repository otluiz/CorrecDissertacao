À meu orientador Prof. Dr. Fernando Buarque, sábio e altivo, sempre soube guiar-me pelos caminhos "não lineares" da pesquisa.

À minha mãe, referência de dedicação e perseverança. Ensinou-me quase tudo que sei, principalmente o gosto pela leitura.

Aos meus filhos, experiência enriquecedora, motivação para fazer melhor e razão para seguir sempre em frente.

A minha amada "Dulcinéia" (Anna Paula), referência de amor e dedicação, interlocutora perspicaz, sempre pronta a ouvir e dialogar. Teve muita paciência com seu cavaleiro errante "Dom Quixote".

À todos da Polícia Rodoviária Federal, em especial ao agente Deivierson,  sempre pronto a esclarecer minhas dúvidas.

À todos os professores da UPE, em especial à Prof. Dra. Maria de Lourdes que com sua dedicação transformaram esta universidade em referência nacional.

Aos todos os colegas de mestrado que se transformaram em melhores amigos, em especial: "Mega", "Rodrigão", "Felipe San", Dupleix, "Pastor Charles", "Fuzzuboy", "Pedro Malandro" e tantos outros que tornaram o ambiente do PPGES alegre, saudável e fecundo em ideias.

À Júlia, profissional dedicada e divertida, quando não falava muito algo estava errado.

Aos colegas da disciplina de Mineração de Dados na UFPE, em especial Orlando e Bruno, tornaram-se grandes amigos e interlocutores para todas as horas.

