%%% Pacotes utilizados %%%

%% Codificação e formatação básica do LaTeX
% Suporte para português (hifenação e caracteres especiais)
%\usepackage[english,brazilian]{babel}

\usepackage[english,brazil]{babel}
%###### USE UM DOS TRÊS: ######%
%*****************************
% Codificação do arquivo
%\usepackage[cp1252]{inputenc}

%\usepackage[portuguese]{babel}

\usepackage[utf8]{inputenc}
%*****************************
% Mapear caracteres especiais no PDF
\usepackage{cmap}

\usepackage{etex}

\usepackage[T1]{fontenc}
\usepackage[brazil]{babel}
\usepackage[dvips]{graphicx}
\usepackage{graphicx}
\usepackage{subfigure}
\usepackage{geometry}


\usepackage{indentfirst}
\usepackage{type 1cm}
\usepackage{lettrine}
\usepackage{colordvi}
\usepackage{colortbl}
\usepackage{multirow}

\usepackage{pstricks}
\usepackage{pstricks-add}
\usepackage{array}

\usepackage{yfonts}
\usepackage{epsf}
\usepackage{wrapfig}
\usepackage{xtab}
\usepackage{eucal}
\usepackage{mathrsfs}

% Essencial para colocar funções e outros símbolos matemáticos
\usepackage{amsmath,amssymb,amsfonts,textcomp}
\usepackage{footmisc}
\usepackage{ifthen}
\usepackage{float}

%Extra
% escrevendo algoritmos
\usepackage[portuguese,ruled,linesnumbered]{algorithm2e}

%Extra
%\usepackage{calligra}                           % Fonte caligr\'{a}fica
%\usepackage{suetterl}                           % Fonte caligr\'{a}fica
\usepackage[utopia]{quotchap}                   % Formata\c{c}\~{a}o para cap\'{\i}tulos
\usepackage{IEEEtrantools_mod}
\usepackage{lscape}

%% Lista de Abreviações
% Cria lista de abreviações
\usepackage[notintoc,portuguese]{nomencl}
\makenomenclature


% Adicionar bibliografia, í­ndice e conteúdo na Tabela de conteúdo
% Não inclui lista de tabelas e figuras no í­ndice
\usepackage[nottoc,notlof,notlot]{tocbibind}


% Conta o número de páginas
\usepackage{lastpage}
\usepackage{setspace}

\pagestyle{plain}


%######### citações modelo abnt 2 ############
\usepackage[num]{abntex2cite}

\usepackage{color, colortbl, multirow}
\usepackage{amssymb}

%###### Margens ######%

\geometry{a4paper,left=3cm,right=2cm,top=3cm,bottom=2cm}

% Comando para capitularização
\usepackage{lettrine}

% Comando para hepígrafe
\usepackage{epigraph}

% Comando para centralizar parágrafos
\usepackage{ragged2e}

%Comando para definir intervalo de parágrafos
\usepackage{lipsum}
%% Comandos customizados

% Espécie e abreviação
\newcommand{\subde}{\emph{Clypeaster subdepressus}}
\newcommand{\subsus}{\emph{C.~subdepressus}}

%utilizando GIF no Latex
\usepackage{epstopdf}

%\epstopdfDeclareGraphicsRule{.gif}{png}{.png}{convert gif:#1 png:\OutputFile}
%\AppendGraphicsExtensions{.gif}


% Título do projeto
%\newcommand{\titulo}{ALGORÍTMO DE RECUPERAÇÃO DA INFORMAÇÃO PARA GRANDES ESPAÇOS DE BUSCA INSPIRADOS EM TÉCNICAS DE BUSCA POR CARDUME DE PEIXES}

\newcommand{\nomedoaluno}{Othon Luiz Teixeira de Oliveira}

\newcommand{\dataQualif}{\vspace{18pt}{Recife, 20 Abril de 2017.}}

