\chapter{Problema, Motivação e Objetivos}\label{intro}

\section{ Problema}\label{intro:problem}

A partir do início do século XXI o mundo digital, especialmente a Internet, conheceu sua primeira grande crise \cite{Quadros2005}.
As empresas ligadas a esse mundo, conhecidas como PontoCom, para sobreviverem, adaptaram-se à Internet abrindo suas estruturas, desde então houve um \textit{boom} nesse 
segmento.
As informações geradas e disponibilizadas à Internet, nos mais recentes anos, representam 90\% de tudo o que já foi criado nos anos anteriores pela humanidade ou desde que 
nossa civilização aprendeu a guardar informação.

Para armazená-los, seriam necessários milhões de computadores; se fosse possível dispô-los num único \textit{DataCenter}, esses ocupariam 
uma área do tamanho do estado de São Paulo.

Os dados produzidos pelo ser humano atualmente dobram a cada 5 anos; esses dados são desde artigos publicados, novas técnicas para os mais diversos problemas da vida humana 
e outros, ficando impossível serem armazenados pelo cérebro humano \cite{bigdataMedicina}.

Com a chegada da Internet das Coisas (IoC), acrônimo de \textit{Internet of Things} (IoT), a previsão é de que o número de informações dobre a cada 2 anos.

A Internet da Coisas pode ser entendida como ``coisas conectadas às coisas'', que pode ser, o refrigerador comunicar-se com os alimentos ali depositados, que contenham uma 
etiqueta identificadora por rádiofrequência (Radio Frequency IDentification -- RFID), podendo ter autonomia para enviar a um supermercado um rol de compras futuras. 

Isso irá fazer com que os dados trafegados pela Internet tenham um crescimento exponencial. A essas informações circulantes dá-se o nome de \textit{Big Data}. 
Um \textit{Big data} é um conceito, na verdade são as coleções de tudo o que é disponibilizado na Internet, desde os dados dos \textit{Data Centers} aos dados gerados pela Internet das Coisas.
Essa enormidade de dados poderia ver a ser um problema de difícil solução se não houvessem abordagens computacionais com habilidade de 
extrair semântica bem como possibilidade de oferecer suporte a agentes decisores.

Uma instância do problema a ser tratado nessa pesquisa será a integração de bases heterogêneas de dados em um aplicação de suporte à decisão de logística de cargas rodoviárias.

%\pagebreak

\section{ Motivação}\label{intro:motivacao}

As rodovias federais que atravessam a Região Metropolitana do Recife (RMR) estão constantemente congestionadas, não apenas pela 
quantidade de veículos, mas por serem alvo de paralisações das mais diversas matizes, como protestos de trabalhadores, acidentes, 
buracos, intempéries naturais e outros tipos de paralisações. 
Em situações extremas poderiam paralisar até a produção das fábricas no seu entorno, por exemplo a Fiat - FCA \cite{BNDES2013}. 
Esta será responsável por aproximadamente 1 000 caminhões cegonheiros nessas rodovias, quando do seu pico de produção (200 000 veículos/ano).

A RMR é a 5ª região mais populosa do Brasil, concentra 3.690.485 habitantes (dados de 2012) \cite{Bitoun2012} em 14 municípios, além da RMR 
será considerada para a pesquisa a Zona da Mata Norte (ZMN) com 577.191 habitantes e a Zona da Mata Sul (ZMS) com 733.447 habitantes. 
Nessas regiões (RMR, ZMN e ZMS) a frota (automóveis particulares, ônibus, caminhões, motocicletas, tratores e outros veículos) 
foi contabilizada, em 2014, com 635.686 veículos \cite{FrotaVeiculosIBGE}.

O que acontece na região metropolitana do Recife, é frequentemente visto no entorno das principais cidades brasileiras.
Por outro lado, câmeras de monitoramento de trânsito, redes sociais, aplicativos de celular e outros dispositivos, fornecem informações diárias sobre o que acontece nessas 
rodovias e no entorno delas, atualizando e alimentando bases de dados históricas, em repositórios espalhados pelos centros de monitoramento de trânsito.

A proposição de uma solução para resolver esse problemática requer várias etapas, para além da proposição de algumas técnicas de mineração dos dados.
Propomos, nesse projeto, uma solução peculiar, ao enviar essa frota de caminhões por diversas rotas, escolhidas por critérios cientificamente estudados.
Isso poderá ser de suma importância para solucionar a problemática do transporte de cargas, que poderá advir com a plena produção da FCA, e que não se aplica apenas 
à região metropolitana do Recife, mas a toda e qualquer fábrica do país. Permitirá fornecer toda informação que se faz necessária para acompanhar os
caminhões na transposição dos obstáculos que possam surgir ao transitar por Pernambuco, conduzindo-os até seu destino de maneira segura e no menor tempo possível.


\section{ Objetivo Geral}\label{intro:objetivo}

Esse projeto de pesquisa tem como objetivo desenvolver um Modelo de sistema de Suporte à Decisão para a problemática das retenções crescentes 
de logística de cargas rodoviárias. Para isto propomos uma solução multidisciplinar através da integração de diversas tecnologias disponíveis desde a análise dos 
dados históricos das rodovias até a situção cotidiana.

\subsection{ Objetivos Específicos}\label{intro:especificos}

\begin{itemize}
 \item Representar a problemática da logística de cargas em uma plataforma adaptável:
 
      \begin{enumerate}
       \item Desenvolver uma plataforma adaptável que utilize dados de redes sociais (Twitter); analisar o contexto das rodovias por essas 
       redes através da mineração de dados em textos e integrar ao resto do sistema através que verifica palavras chaves como: protestos, acidentes e outras.
      \end{enumerate}

 \item Desenvolver um modelo preditivo dos fenômenos que envolvem as rodovias:

      \begin{enumerate}
       \item Para o desenvolvimento do modelo preditivo pretende-se utilizar bases de dados históricas.
       \item O modelo preditivo integra várias bases de dados, tais como: Polícia Rodoviária Federal -- PRF, Batalhão de Polícia de Transito -- BPRv e dados históricos 
       do Instituto de Pesquisas Espaciais -- INPE ou dos dados de precipitações pluviométricas do ``European Centre for Medium-Range Weather -- ECMWF.
       \item Algumas dessas informações também estão disponíveis em base de dados abertas, como sugere o Portal da Transparência, nos servidores da PRF além de outras informações para complementar o sistema estão 
       disponíveis na Internet sendo atualizadas pela PRF através de uma API aberta, esta pode ser configurável para se ligar ao nosso sistema.
      \end{enumerate}

 \item Propor uma simulação interativa da estrutura com a dinâmica:
      
      \begin{enumerate}
       \item A malha viária está representada na Internet em mapas de bases vetoriais que pretendemos integrar ao nosso sistema de informações e também vamos incorporar 
       as informações que a Polícia Rodoviária Federal(PRF) dispõe, que controla as rodovias BRs. 
       \item A simulação interativa utiliza uma plataforma baseada na API do Google Maps.
       \item Para simulação interativa da estrutura preditiva com a dinâmica real serão capturados ``feeds'' de redes sociais, por exemplo pelo Twitter. Essa técnica 
       fará um arco cibernético mantendo o sistema preditivo atualizado.
      \end{enumerate}

\end{itemize}


\section{ Resultados Esperados}\label{resultado}

Ao final dessa pesquisa pretende-se obter um Modelo de Sistema de Suporte à Decisão adaptável, que contribua como uma ferramenta para  
complementar à tomada de decisão para a gestão do transporte de cargas rodoviárias. A região inicialmente escolhida será a RMR.








