\chapter{Problema, Motivação e Objetivo}\label{intro}

\section{Problema}\label{intro:problem}

A partir do início do século XXI o mundo digital, especialmente a Internet, conheceu sua primeira grande crise \cite{Quadros2005}.
As empresas ligadas a esse mundo, conhecidas como PontoCom, para sobreviverem, adaptaram-se à Internet abrindo suas estruturas, desde então houve um \textit{boom} nesse 
segmento.
As informações geradas e disponibilizadas à Internet, nos mais recentes anos, representam 90\% de tudo o que já foi criado nos anos anteriores pela humanidade ou desde que 
nossa civilização aprendeu a guardar informação.

Para armazená-los, seriam necessários milhões de computadores; se fosse possível dispô-los num único \textit{DataCenter}, esses ocupariam 
uma área do tamanho do estado de São Paulo.

Os dados produzidos pelo ser humano atualmente dobram a cada 5 anos; esses dados são desde artigos publicados, novas técnicas para os mais diversos problemas da vida humana 
e outros, ficando impossível serem armazenados pelo cérebro humano \cite{bigdataMedicina}.

Com a chegada da Internet das Coisas (IoC), acrônimo de \textit{Internet of Things} (IoT), a previsão é de que o número de informações dobre a cada 2 anos.

A Internet da Coisas pode ser entendida como ``coisas conectadas às coisas'', que pode ser, o refrigerador comunicar-se com os alimentos ali depositados, que contenham uma 
etiqueta identificadora por rádiofrequência (Radio Frequency IDentification -- RFID), podendo ter autonomia para enviar a um supermercado um rol de compras futuras. 

Isso irá fazer com que os dados trafegados pela Internet tenham um crescimento exponencial. A essas informações circulantes dá-se o nome de \textit{Big Data}. 
Um \textit{Big data} é um conceito, na verdade são as coleções de tudo o que é disponibilizado na Internet, desde os dados dos \textit{Data Centers} aos dados gerados pela Internet das Coisas.
Essa enormidade de dados poderia ver a ser um problema de difícil solução se não houvessem abordagens computacionais com habilidade de 
extrair semântica bem como possibilidade de oferecer suporte a agentes decisores.

Uma instância do problema a ser tratado nessa pesquisa será a integração de bases heterogêneas de dados em um aplicação de suporte à decisão de logística de cargas rodoviárias.

%\pagebreak

\section{Motivação}\label{intro:motivacao}

As rodovias federais que atravessam a Região Metropolitana do Recife (RMR) estão constantemente congestionadas, não apenas pela 
quantidade de veículos, mas por serem alvo de paralisações das mais diversas matizes, como protestos de trabalhadores, acidentes, 
buracos, intempéries naturais e outros tipos de paralisações. 
Em situações extremas poderiam paralisar até a produção das fábricas no seu entorno, por exemplo a Fiat - FCA \cite{BNDES2013}. 
Esta será responsável por aproximadamente 1 000 caminhões cegonheiros nessas rodovias, quando do seu pico de produção (200 000 veículos/ano).

A RMR é a 5ª região mais populosa do Brasil, concentra 3.690.485 habitantes (dados de 2012) \cite{Bitoun2012} em 14 municípios, além da RMR 
será considerada para a pesquisa a Zona da Mata Norte (ZMN) com 577.191 habitantes e a Zona da Mata Sul (ZMS) com 733.447 habitantes. 
Nessas regiões (RMR, ZMN e ZMS) a frota (automóveis particulares, ônibus, caminhões, motocicletas, tratores e outros veículos) 
foi contabilizada, em 2014, com 635.686 veículos \cite{FrotaVeiculosIBGE}.

O que acontece na região metropolitana do Recife, é frequentemente visto no entorno das principais cidades brasileiras.
Por outro lado, câmeras de monitoramento de trânsito, redes sociais, aplicativos de celular e outros dispositivos, fornecem informações diárias sobre o que acontece nessas 
rodovias e no entorno delas, atualizando e alimentando bases de dados históricas, em repositórios espalhados pelos centros de monitoramento de trânsito.

A proposição de uma solução para resolver esse problemática requer várias etapas, para além da proposição de algumas técnicas de mineração dos dados.
Propomos, nesse projeto, uma solução peculiar, ao enviar essa frota de caminhões por diversas rotas, escolhidas por critérios cientificamente estudados.
Isso poderá ser de suma importância para solucionar a problemática do transporte de cargas, que poderá advir com a plena produção da FCA, e que não se aplica apenas 
à região metropolitana do Recife, mas a toda e qualquer fábrica do país. Permitirá fornecer toda informação que se faz necessária para acompanhar os
caminhões na transposição dos obstáculos que possam surgir ao transitar por Pernambuco, conduzindo-os até seu destino de maneira segura e no menor tempo possível.


\section{Objetivo Geral}\label{intro:objetivo}

Esse projeto de pesquisa tem como objetivo desenvolver um sistema de Suporte à Decisão para a problemática das retenções crescentes 
de logística de cargas rodoviárias. Para isto propomos uma solução multidisciplinar através da integração de diversas tecnologias disponíveis desde a análise dos 
dados históricos das rodovias até a situção cotidiana.

\subsection{Objetivos Específicos}\label{intro:especificos}

\begin{itemize}
 \item Representar a problemática da logística de cargas em uma plataforma adaptável:
      \begin{itemize}
       \item Está sendo desenvolvido uma plataforma adaptável que integre dados de redes sociais (Twitter) que analisa o contexto das rodoviárias através da mineração
       de textos que verifica palavras chaves como: protestos, acidentes e outras.
      \end{itemize}

 \item Desenvolver um modelo preditivo dos fenômenos que envolvem as rodovias.
      \begin{itemize}
       \item O modelo preditivo integra várias bases de dados, tais como: Polícia Rodoviária Federal -- PRF, Batalhão de Polícia de Transito -- BPRv e dados históricos 
       do Intitudo de Pesquisas Espaciais -- INPE.
      \end{itemize}

 \item Propor uma simulação interativa da estrutura com a dinâmica.
      \begin{itemize}
       \item A simulação interativa é uma plataforma baseada na API do Google Maps
      \end{itemize}

\end{itemize}



\section{Reflexão sobre as tecnologias}\label{result}

Não existe uma técnica de mineração que generalise os mais diversos ambientes preditivos, mas sim um ``pool'' dessas técnicas onde uma complementa outra.
As técnicas preditivas tradicionais que contemplam análise de grandes massas de dados como base heterogêneas são possíveis quando adaptadas para uma forma comparável à que
foram inicialmente concebidas. Algumas técnicas de IA são altamente sensíveis a dados ausentes os ``missing data'' à dados com pouca consistência e outros tipos de dados 
comuns em bases mantidas sem um bom critério de inserção dos dados. Nesta pesquisa, bases heterogêneas foram integralizadas num única grande base, onde as variáveis independentes foram
em sua maioria preservadas e/ou construídas novas, nas bases onde não haviam correspondência, respeitando a lógica do negócio. 
A variável dependente foi designada como \textbf{gargalo} e as variáveis independentes (ou explicativas) são: \\

\hspace{1cm}

\begin{tabular}{c  l}
 KM & \textit{Numeração do Quilômetro}\\
 BR & \textit{Numeração da BR}\\
 condPista & \textit{Condição da Pista: Seca, Molhado, ...}\\
 restVisibili & \textit{Restrição de Visibilidade: Inexistente, Neblina, .., outros}\\
 tipoAcident & \textit{Tipo de Acidente: Atropelamento, Colisão, Paralisação,...}\\
 tipoDano  & \textit{Tipo de Dano: Leve, Médio, Grave}\\
 Municipio  & \textit{Localidade onde ocorreu}\\
 Ano & \textit{Ano que o acidente ocorreu}\\
 Mês & \textit{Mês que o acidente ocorreu}\\
 Dia & \textit{Dia que o acidente ocorreu}\\
 Hora & \textit{Hora que o acidente ocorreu}\\
\end{tabular}

\hspace{1cm}
 
A base de dados da PRF, relativas a paralisações das vias, por motivos diversos, não haviam variáveis tais como (visibilidade, condições da via, gravidade paralisação e outras).
Presumivelmente protestos são realizados com boa visibiliade, em condições de via rezoáveis e a gravidade da paralisação foi considerada leve.

As técnicas como Redes Neurais Artificias (MLP), Árvores de decisão (CART), Regressão logística (MLR) nos forneçam uma
visão generalizada dos fatores preponderantes quando os dados estão tratados.

\begin{itemize}
 \item[a] Redes Neurais Artificias do tipo Multi Layer Perceptron -- (MLP) têm capacidade de receber várias entradas ao mesmo tempo e distribuí-las de maneira organizada, além 
 são simples de implementar e trazem resultados satisfatórios em grandes bases de dados.
 
 \item[b] Árvores de decisão para classificar acidentes do tipo Classification and Regression Tree -- (CART) foi empregue por Pakgohar et al no artigo ''The role of human factor in incident and
severity of road crashes based on the CART and LR regression a data mining approach`` com nível de acurácia próximo aos 80\%

  \item[c] Regressão logística tipo Multinomial Logistic Regression -- (MLR) fornece a possíbilidade de aprofundamento em vários níveis de busca sendo a mais apropriada, já que Regressão logistica
não permite aprofundamento no espaço de busca.


\end{itemize}




