\chapter{Introdução}\label{intro}

\section{Justificativa do problema}\label{intro:problem}

O século XXI caracteriza-se como sendo a Era do crescimento exponencial da informação. Essas informações são produzidas tanto por seres humanos, quanto por máquinas. Segundo Nobert Wiener \cite{Salles2007}, a informação tem tanta importância quanto a energia e a matéria. Essa informação pode ser utilizada para controlar sistemas baseados em comportamento biológico ou mecânico. Esse comportamento, quando controlado por meio de realimentação, tem como alvo atingir um objetivo, um propósito, como compreender, controlar, predizer.

Os dados produzidos pelo ser humano atualmente dobram a cada cinco anos. As redes sociais, muito mais do que um ambiente lúdico, se configuram como um espaço onde as pessoas vão buscar informações para a gestão dos seus problemas cotidianos, bem como um lugar de coleta de informações para sistemas inteligentes proporem soluções mais adequadas á problemática humana e, ao mesmo tempo, com rapidez.

Dados governamentais têm sido disponibilizados pelo governo brasileiro desde que este aderiu em 2009 ao movimento mundial para incentivar as autoridades dos países a maior transparência e participação popular conhecido, este movimento foi conhecido como ``Open Data'' \cite{DadosGoverno}. Desde então o Brasil vem se esforçando para disponibilizar informações governamentais para todos os cidadãos. Os dados utilizados nesta pesquisa também podem ser encontrados pelo Sistema BR-Brasil, da Polícia Rodoviária Federal. A Polícia Rodoviária Federal regista diariamente boletins de ocorrência, contudo, dados produzidos eletronicamente só estão disponíveis a partir de 2007.

A inteligência artificial é uma área vai buscar essas informações e, com algoritmos eficientes, propor soluções inteligentes para dar conta das mais diversas necessidades humanas, sobretudo aquelas relacionadas ao contexto social, como logística de transporte, locomoção de pessoas, gestão de tempo, dentre outros. A Mineração de Dados (MD) vai buscar a Inteligência Artificial algoritmos para descoberta de padrões e automatizar tarefas na investigação dos dados, essa automatização também conhecida como ``Machine Learning'' aplica-se a quase todos os caminhos na descoberta do conhecimento oferecida pela MD.
 
Uma instância do problemática descrita acima será tratada nesta pesquisa: o tráfego de veículos, transporte de mercadorias e locomoção nas rodovias.  Para isso será necessária a integração de bases de dados heterogêneas disponíveis em computadores de órgãos públicos 
que contenham informações de qualidade para gerar um modelo preditivo de roteamento logístico de transporte. Para isso serão considerados dados históricos de cada rodovia, com os trechos onde há mais 
retenções que causam constrangimento nessas vias em determinados períodos do dia, que se repetem em meses e ao longo dos anos, tais como acidentes, protestos, intempéries ambientais.
De forma complementar, serão utilizados informações de redes sociais, como o Twitter. A escolha dessa rede social se deu pelo fato de que um dos seus principais objetivos é o de compartilhar informações sucintas e pontuais entre os seus usuários, boa parte delas sobre eventos que influenciam o cotidiano das pessoas.

Esta dissertação está organizada da seguinte forma:

No Capítulo 1 a Introdução ...

No Capítulo 2 é a Revisão da Literatura ...

No Capítulo 3 está nossa Contribuição ...

No Capítulo 4 a Simulação e execução do modelo proposto ...

No Capítulo 5 nossas Considerações Finais ...

Os anexos trazem os dados originais e ....

%\pagebreak

\section{ Motivação}\label{intro:motivacao}

As rodovias federais que atravessam a região metropolitana e interior do estado de Pernambuco estão constantemente congestionadas, não apenas pela 
quantidade de veículos, mas por serem alvo de paralisações das mais diversas matizes, como protestos de trabalhadores, acidentes, danos na via, intempéries naturais e outros tipos de constrangimentos que interferem no fluxo de veículos. 
Em situações extremas poderiam paralisar até a produção das fábricas no seu entorno \cite{BNDES2013}. 

A RMR é a 5ª região mais populosa do Brasil, concentra 3.690.485 habitantes (dados de 2012) em 14 municípios, além da 
Zona da Mata Norte (ZMN) com 577.191 habitantes e a Zona da Mata Sul (ZMS) com 733.447 habitantes \cite{Bitoun2012}. 
Nessas regiões (RMR, ZMN e ZMS) a frota (automóveis particulares, ônibus, caminhões, motocicletas, tratores e outros veículos) 
foi contabilizada, em 2015, com mais de 1.270.000 veículos \cite{FrotaVeiculosIBGE}. Se considerarmos o interior do estado, essa frota aumenta para mais de 2.700.000 veículos, distribuídos nas regiões do Agreste e Sertão. Algumas cidades se destacam por concentrarem uma frota maior, como Caruaru, no agreste pernambucano, com mais de 150.000 veículos, e Petrolina, no sertão, com quase 130.000. 
																			
O que acontece nas grandes cidades do estado de Pernambuco e no seu entorno é frequentemente visto nas grandes cidades brasileiras.
Por outro lado, câmeras de monitoramento de trânsito, redes sociais, aplicativos de celular e outros dispositivos, fornecem informações diárias sobre o que acontece nessas 
rodovias e no entorno delas, atualizando e alimentando bases de dados históricas, em repositórios espalhados pelos centros de 
monitoramento de trânsito, isso é conhecido como \textit{Big data}.

Fora do perímetro urbano as rodovias atravessam outras localidades com problemáticas diversas, tais como pavimento ruim ou ausência de pavimentação, 
traçados inapropriados e outras intempéries têm causado frequentemente acidentes.
A Polícia Rodoviária Federal ou outros órgãos de controle público atendem e registram esses acontecimentos em boletins diários.

A proposição de uma solução para absorver parte dessas informações requer várias etapas, que engloba algumas técnicas de mineração de dados.
Propomos, nessa pesquisa, uma solução peculiar para utilização das rotas existentes, definida por critérios cientificamente estudados, que seja materializado num modelo de predição.
Isso poderá ser de suma importância para solucionar a problemática do tráfego em rodovias, fornecendo toda informação que se faz necessária para que o veículo até seu destino de maneira segura e no menor tempo possível.

%\pagebreak

\section{ Objetivo Geral}\label{intro:objetivo}

Essa pesquisa teve como objetivo principal desenvolver um modelo preditivo de suporte à decisão para a problemática das retenções crescentes nas rodovias pernambucanas. 
Para isso, propomos uma solução multidisciplinar através da integração de diversas tecnologias disponíveis,que vão desde a análise dos dados históricos das 
rodovias à utilização informações de redes sociais e dados governamentais.

\subsection{ Objetivos Específicos}\label{intro:especificos}

\begin{itemize}
 \item Caracterizar a problemática de cada rodovia; 
 \item Desenvolver um modelo preditivo dos fenômenos que envolvem as rodovias;
 \item Desenvolver um ambiente de simulações interativas da estrutura com a dinâmica.
 \item Propor soluções para melhor experiência dos usuários que utilizam as rodovias pernambucanas.
\end{itemize}










