\chapter{Introdução}\label{intro}

\section{Justificativa do problema}\label{intro:problem}

A partir do início do século XXI o mundo digital, especialmente a Internet, conheceu sua primeira grande crise \cite{Quadros2005}.
As empresas ligadas a esse mundo, conhecidas como PontoCom, para sobreviverem, adaptaram-se à Internet abrindo suas estruturas.
Desde então houve um \textit{boom} de informações disponíveis nesse segmento sócio-econômico.
As informações geradas e disponibilizadas à Internet, nos mais recentes anos, representam 90\% de tudo o que já foi criado nos anos 
anteriores pela humanidade, ou desde que nossa civilização aprendeu a guardar informação.

Para armazená-los, seriam necessários milhões de computadores. 
Caso fosse possível dispô-los num único \textit{DataCenter}, esses ocupariam uma área do tamanho do estado de São Paulo.

Os dados produzidos pelo ser humano atualmente dobram a cada 5 anos; astrônomos atualizam suas descobertas numa base de dados disponíveis para outros utilizarem; as ciências biológicas agora têm tradição 
em depositar seus avanços científicos em repositórios públicos \cite{bigdataMedicina}; redes sociais estão focadas na Web: Facebook, LinkedIn, Tweeter e outras 
sobrevivem coletando informações, vendendo espaço publicitário e repassando-as às empresas de telemarketing; empresas de comércio eletrônico como 
Amazon.com, Submarino.com.br, Americanas.com, MagazineLuíza.com.br, utilizam essas informações para vender mais e melhor. 
Por outro lado, artigos científicos dos mais variados assuntos e das mais variadas áreas alimentam todos os dias, com milhões de 
informações, os \textit{Data Centers}.

Uma instância do problema descrito anteriormente será tratado nesta pesquisa. 
Para isso será necessária a integração de bases de dados heterogêneas disponíveis em computadores de órgãos públicos 
que contenham informações de qualidade para gerar um modelo preditivo de roteamento logístico de cargas rodoviárias, considerando dados históricos de cada rodovia, com os trechos onde há mais 
retenções que causam constrangimento nessas vias em determinados períodos do dia, que se repetem em meses e ao longo dos ano, tais como acidentes, protestos, intempéries ambienteais.
Associadas ao problema em lide, as redes sociais são um arcabouço de informações, os utilizadores dessas redes fornecem uma grande quantidade de dados que podem ser filtrados para 
dentro de uma aplicação, através de técnicas adequadas.

%\pagebreak

\section{ Motivação}\label{intro:motivacao}

As rodovias federais que atravessam a Região Metropolitana do Recife (RMR) estão constantemente congestionadas, não apenas pela 
quantidade de veículos, mas por serem alvo de paralisações das mais diversas matizes, como protestos de trabalhadores, acidentes, 
buracos, intempéries naturais e outros tipos de paralisações. 
Em situações extremas poderiam paralisar até a produção das fábricas no seu entorno \cite{BNDES2013}. 

A RMR é a 5ª região mais populosa do Brasil, concentra 3.690.485 habitantes (dados de 2012) \cite{Bitoun2012} em 14 municípios, além da 
Zona da Mata Norte (ZMN) com 577.191 habitantes e a Zona da Mata Sul (ZMS) com 733.447 habitantes. 
Nessas regiões (RMR, ZMN e ZMS) a frota (automóveis particulares, ônibus, caminhões, motocicletas, tratores e outros veículos) 
foi contabilizada, em 2014, com 635.686 veículos \cite{FrotaVeiculosIBGE}.

O que acontece na região metropolitana do Recife e no seu entorno é frequentemente visto nas grandes cidades brasileiras.
Por outro lado, câmeras de monitoramento de trânsito, redes sociais, aplicativos de celular e outros dispositivos, fornecem informações diárias sobre o que acontece nessas 
rodovias e no entorno delas, atualizando e alimentando bases de dados históricas, em repositórios espalhados pelos centros de 
monitoramento de trânsito, isso é conhecido como \textit{Big data}.

Fora do perímetro urbano as rodovias atravessam outras localidades com problemáticas diversas tais como pavimento ruim ou mesmo sem pavimentação, 
traçados inapropriados e outras intempéries têm causado frequentemente acidentes.
A Polícia Rodoviária Federal ou outros órgãos de controle público atendem e registram esses acontecimentos em boletins diários.

A proposição de uma solução para absorver parte dessas informações requer várias etapas, para além da proposição de algumas técnicas de mineração dos dados.
Propomos, nesse mestrado, uma solução peculiar, ao enviar essa frota de caminhões por diversas rotas, escolhidas por critérios cientificamente estudados.
Isso poderá ser de suma importância para solucionar a problemática do transporte de carga na região metropolitana 
do Recife. Permitirá fornecer toda informação que se faz necessária para acompanhar veículos de carga, como por exemplo caminhões, 
na transposição dos obstáculos que possam surgir ao transitar por Pernambuco, conduzindo-os até seu destino de maneira segura e no menor tempo possível.


\section{ Objetivo Geral}\label{intro:objetivo}

Esse estudo tem como objetivo principal desenvolver um modelo preditivo de suporte a decisão para a problemática das retenções crescentes de transporte de cargas rodoviárias 
nas BRs pernambucanas, contudo este estudo poderá ser portado para outras regiões brasileiras. 
Para isto propomos uma solução multidisciplinar através da integração de diversas tecnologias disponíveis desde a análise dos dados históricos das 
rodovias a utilização informações de redes sociais e dados governamentais.

\subsection{ Objetivos Específicos}\label{intro:especificos}

\begin{itemize}
 \item Representar a problemática da logística de cargas em uma plataforma adaptável;

 \item Desenvolver um modelo preditivo dos fenômenos que envolvem as rodovias;

 \item Desenvolver um ambiente de simulações interativa da estrutura com a dinâmica.

\end{itemize}


\section{ Resultados Esperados}\label{resultado}

Ao final dessa pesquisa pretende-se obter um Modelo de Sistema de Suporte à Decisão adaptável, que contribua como uma ferramenta para  
complementar à tomada de decisão para a gestão do transporte de cargas rodoviárias. A região inicialmente escolhida será a RMR.








