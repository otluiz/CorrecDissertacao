\chapter{Contribuição}\label{meto}

A contribuição dessa pesquisa é de cunho metodológico-prático.
Do ponto de vista metodológico pela aplicação do processo CRISP-DM, usado para construir o modelo preditivo; do ponto de vista prático 
pela proposição de um modelo que integre predição à API de mapas de posicionamento global, fornecendo informação suficiente a um gestor para decidir quando 
e por onde enviar uma frota de caminhões por determinada rodovia que apresente retenções crescentes de logística de cargas. 

As soluções disponíveis que existem tais como; Google Maps, Waze e outros dessa natureza somente exibem informações momentâneas, produzidas e compartilhadas pelos utilizadores 
dos aplicativos ou por informações provindas de GPS, contudo não analisam dados históricos dessas rodovias nem fazem predições sobre o seu comportamento.

Outra contribuição dessa pesquisa é a proposição de um arco cibernético construído com a API de redes sociais.
Os ``feeds'' de notícias das redes sociais como o Twitter permitem analisar o contexto das rodovias com defasagem temporal muito pequena.
Os utilizadores dessas redes sociais contribuem com muita informação relevante como por exemplo o anúncio de uma paralisação que ocorrerá 
daqui a uma semana, a PRF de Pernambuco é outro contribuidor permanente; com seu canal no Twitter: @PRF191PE fornece diariamente informação das rodovias 
além de dados estatísticos. 

A monitoração de redes sociais é feita por Mineração de dados em textos, em que são verificas palavras chaves como: protestos, acidentes e outras.
Uma vez capturadas e tratadas, as informações desses ``feeds'' são direcionadas à um banco de dados onde será possível confrontar com o 
modelo preditivo aumentando o nível de confiança, por exemplo: no trecho da Br 101, na altura do km 5, no Município de Goiana alguém publicou
que a comunidade que mora no entorno dessa localidade fará um protesto daqui a dois dias devido ao acidente ocorrido ou a PRF publicou que o km 80
da Br 232, na altura do Município de Gravatá será interditada amanhã, por 2h, para remoção/explosão de rochas. 
Essas informações, por serem a posteriori às predições, podem aumentar o nível de confiança do utilizador e dentro de um universo temporal 
mais restrito servir de comprovação do modelo proposto.
As informações de redes sociais, armazenadas em um banco de dados, poderão servir futuramente para novas predições.

\section{Modelo Proposto}

A metodologia utilizada nessa pesquisa contempla um plano em três etapas, cada uma dividida em fases atinentes.
A primeira etapa da nossa metodologia completam o ciclo do processo CRISP-DM, onde está o modelo preditivo e 
a descoberta de conhecimento sobre o comportamento das rodovias. O descoberta de conhecimento sobre esses comportamento 
nas rodovias tem a ver com o ``modus operandi'' dos utilizadores, sobre possíveis erros de traçados e outros que possam
ser identificados pelos algoritmos de mineração empregados no processo.

A priori foram escolhidos algoritmos com algumas características especiais, tais como; robustez, tolerância à faltas (missing data),
aprendizagem taxa de aprendizagem, e facilidade de interpretação dos dados processados. 
No quesito robustez, tolerância à faltas e taxa de aprendizagem as redes neurais artificiais (RNA), com uma 
topologia relativamente grande destacam-se pela capacidade de generalização, contudo como todos modelos estatísticos pode ocorrer 
superadaptação (overfitting), segundo RUSSEL E NORVIG (2004)  \footnote{Foi observado que redes neurais muito grandes \textit{generalizam} bem, 
\textit{desde que os pesos sejam mantidos pequenos}. Essa restrição mantém os valores de ativação na região 
\textit{linear} da função sigmóide g(x) onde x é próximo de zero. Por sua vez isso faz com que a rede se comporte 
como uma função linear, com um número muito menor de parâmetros.} o ``overfitting''
 ocorre quando o número atributos é grande.

A extrapolação do modelo preditivo ocorre quando este se integra a um estrutura dinâmica a serem exibidas em mapas vetoriais, 
dado um espaço temporal pré-determinado por um agente; o utilizador. 
Através de APIs os mapas vetoriais permitem a geolocalização dos pontos classificados ou os pontos onde haverá grande número de 
retenções, conhecido no meio da logística de cargas como \textbf{gargalo}.
A API do Google-Maps é o ``front end'', foi escolhida por permitir maior pela portabilidade e simplicidade para integração da estrutura
dinâmica com a preditiva.

Para a integração às redes sociais, foi escolhida a API do Twitter. Esta ``interface'' é simples de ser configurada e gerar poucos dados; 
o utilizador tem que ser eficaz ao gerar suas postagem em um espaço de 140 caracteres, isso facilita a forma como os dados são
extraídos pela quantidade diminuta deles, bem como a quantidade conexões à Internet, contudo está rede social tem uma crescente quantidade
de postagens no formato imagens, isso dificulta a mineração em textos.
A API do Twitter tem a finalidade de integrar o modelo dinâmica dos mapas vetoriais às redes sociais. 
Esta ``interface'' é responsável por fornecer ``input'' à terceira etapa, servindo de ``busca local'' das informações mais recentes das 
redes sociais, relativas à trechos das rodovias; os ``feeds'' do Twitter (ou twetts) fornecem dados que serão minerados e interpretados à posteriori.

A figura a seguir ilustra (um overview) essa metodologia descrita graficamente.

\begin{figure}[ht]
\centering
\caption{Etapas da modelo proposto}
\includegraphics[width=170mm, height=85mm]{Figuras/Cronograma/metodologiaGeral.png}\\
\tiny Fonte: autor
\end{figure}

\section{Reflexão sobre as tecnologias utilizadas no modelo preditivo}\label{result}

Não existe uma técnica de mineração que generalize os mais diversos ambientes preditivos, mas sim um ``pool'' 
dessas técnicas onde uma complementa outra. ((( citar )))

As técnicas preditivas tradicionais que contemplam análise de grandes massas de dados como base homogêneas.

são possíveis quando adaptadas para uma forma comparável à que
foram inicialmente concebidas, por que as variáveis em uma base de dados a priori guardam pouca relação as variáveis de outra base de dados.
neste caso essas variáveis ou são excluídas ou são transformadas a fim de ``guardarem'' um correlação com a outra base de dados. 

Na fase de transformação de dados, da primeira etapa, onde são criadas novas variáveis, a proximidade entre as
bases heterogêneas foi conseguido utilizando de regras de indução da lógica proposicional. ((( citar ))).
Nesta pesquisa, bases heterogêneas foram integralizadas num única grande conjunto de dados o ``data set''. As variáveis desse ``data set''
são consideradas variáveis independentes, foram preservadas as com maior relevância ou as que continham a maior quantidade de conhecimento
embutido.  construídas novas, nas bases onde não haviam correspondência, respeitando a lógica do negócio.\\
A tabela a seguir descreve as variáveis originais na base de dados de acidentes da PRF 


\section{Extração do conhecimento - KDD}

As técnicas como Redes Neurais Artificias (MLP) \cite{DecisaoCredito}, Árvores de decisão (CART) \cite{DataMining}, Regressão logística (MLR) 
\cite{RegrecaoLog} fornecem visão generalizada dos fatores preponderantes, levantando padrões ocultos nos dados. Esta fase é conhecida como 
Aprendizagem de Máquina (acrônimo de Machine Learning)

\begin{itemize}
 \item[a] Redes Neurais Artificias do tipo \textit{ Multi Layer Perceptron}  -- (MLP) têm capacidade de receber várias entradas ao mesmo tempo e distribuí-las de maneira organizada, além 
	  são simples de implementar e trazem resultados satisfatórios em grandes bases de dados.
 
 \item[b] Árvores de decisão do tipo \textit{ Classification and Regression Tree}  -- (CART) foi empregue por Pakgohar et al no artigo 
	  \textit{The role of human factor in incident and severity of road crashes based on the CART and LR regression a data mining approach}  para classificar acidentes 
	  com nível de acurácia próximo aos 80\%

 \item[c] Regressão logística tipo \textit{Multinomial Logistic Regression} -- (MLR) fornece a possibilidade de aprofundamento em vários níveis de busca sendo a mais apropriada, já que Regressão logística 
	  tradicional não permite aprofundamento desse tipo no espaço de busca.
\end{itemize}


