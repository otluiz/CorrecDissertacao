\chapter{Simulação}\label{simula}

\section{ As variáveis do modelo preditivo}

Algumas técnicas de IA são altamente sensíveis a dados ausentes os ``missing data'' à dados com pouca consistência e outros tipos de dados 
comuns em bases mantidas sem um bom critério de inserção dos dados. 
A variável dependente foi designada como \textbf{gargalo} e as variáveis independentes (ou explicativas) são:


\begin{table}[htbp]
 \centering
  \caption{Variáveis do modelo preditivo}
  
  \begin{tabular}{r|l} \hline
   KM & Numeração do quilômetro \\
   BR & Numeração da Br\\
   condPista & Condição da pista: seca, molhado, ... \\
   restVisibili & Restrição de visibilidade: inexistente, neblina, .., outros \\
   tipoAcident & Tipo de Acidente: atropelamento, colisão, paralisação,...\\
   tipoDano  & Tipo de Dano: leve, médio, grave \\
   Municipio  & Localidade onde ocorreu \\
   Ano & Ano que ocorreu o acidente \\
   Mês & Mês que ocorreu o acidente \\
   Dia & Dia que ocorreu o acidente \\
   Hora & Hora que ocorreu o acidente \\
  \end{tabular}
\end{table}

 
À base de dados da PRF, relativas a interdições das vias, por motivos diversos, não haviam variáveis tais como; visibilidade, condições da via, gravidade paralisação e outras.
Foram incorporadas à essa base essas novas variáveis, para populá-las, adotou-se a lógica; presumivelmente protestos são realizados com boa visibilidade, em condições de via razoáveis e a gravidade da paralisação foi considerada leve.

\section{Casos de teste}

\subsection{Base da PRF(acidentes) X Base da PRF(interdições)}

\subsection{Base da PRF(interdições) X Base do IBGE(DataSus)}