\chapter{Simulação}\label{simula}

\section{Execução do modelo}

Esse capítulo apresenta os resultados encontrados em cada etapa, no processo de desenvolvimento do modelo de predição de ocorrências nas principais BRs do estado de Pernambuco, para definição de melhores rotas e horários para os usuários dessas vias.

\vspace{5mm}

A figura a seguir resume as três etapas contempladas na proposição do modelo.

\begin{figure}[ht]
\centering
\caption{Etapas da modelo proposto}
\includegraphics[width=165mm, height=100mm]{Figuras/Cronograma/metodologia.png}\\
\tiny Fonte: o Autor
\end{figure}

A \textbf{Etapa 1} - KDD contempla a fase da coleta das bases de dados históricos, preparação dos dados, construção das variáveis do modelo preditivo, descoberta dos pontos críticos das rodovias (que serão utilizados na etapa de georreferencimento);
  \begin{enumerate}
    \item O modelo preditivo integra as bases de dados da Polícia Rodoviária Federal -- PRF.
     Algumas dessas informações também estão disponíveis em base de dados abertas, como sugere o Portal da Transparência, nos servidores da PRF, além de outras informações complementares;
    
    \item Em seguida as bases heterogêneas são concatenados de acordo com a lógica do negócio que se propôs, como por exemplo algumas variáveis podem ser construídas a fim de permitir a concatenação dos dados;
    
    \item Uma vez concatenados (as bases de dados) e limpos (retirados os ``missing data''), os dados serão dividos em conjunto de treinamento e de testes;
   
    \item Uma parte do conjunto de dados, o conjunto de trenamento serve para treinar os diversos algoritmos de IA já previamente escolhidos;
    
    \item Nesta fase os algoritmos de IA são utilizados a fim de produzirem resultados esperados pelas técnicas escolhidas. Pode haver recorrência nesta fase caso os algoritmos não tenham produzidos os resultados esperados. A conclusão dessa etapa ocorre com a Mineração dos dados e a extração de conhecimento.
    
    \item A fase de avaliação dos resultados ocorre pela análise das diversas métricas definidas como: Instâncias corretamente classificadas, Áreas sob a curva ROC (AUC) e Matriz de Confusão.
     
	\item A escolha do melhor resultado se dá pelo maior valor da área encontrada sob a curva ROC concomitantemente com o maior número de instâncias classificadas corretamente e o menor número de instâncias classificadas incorretamente com o menor erro médio absoluto. 
\end{enumerate}
  
  
A \textbf{Etapa 2} - Redes Sociais contempla:
 \begin{enumerate}
 	\item Um módulo dinâmico onde são capturados ``feeds'' da rede social Twitter. 
 	     Essa técnica faz um arco cibernético mantendo o utilizador atualizado com as informações recentes.

 	\item Após a captação dos dados do Twitter é feita a Mineração nos textos para localizar informações que permitam antever alguma paralização futura nas rodovias. 
  \end{enumerate}

A \textbf{Etapa 3} - Mapas é última etapa, que consiste em um módulo com as seguintes características:
  \begin{enumerate}
    \item Conversão dos resultados encontrados na Etapa 1, localizações geográficas dos pontos críticos da malha viária, indicadas pelo Km, são agrupadas formando ''clusters`` de dados exibidos em mapas vetoriais;
    
    \item Um ambiente de simulação interativa que utiliza uma plataforma baseada na API do Google Maps.
    
    \item Exibe informações vindas do twitter sobre ocorrências na rodovia, encontrando sua geolocalização a ser transformado em marcos ``milestone'' para representação sobre mapas de bases vetoriais.
  \end{enumerate}
 
\pagebreak 

\section{A construção do Modelo preditivo}

O modelo preditivo foi construído utilizando bases de dados históricos da PRF (de acidentes e de interdições, por exemplo: protestos) entre Janeiro de 2007 a Dezembro 2015. Cada ano correspondia a uma base de dados independente, tendo sido integradas, formando uma base única com aproximadamente 85 mil registros. 


\subsection{Aplicação do CRISP-DM}
O CRISP-DM nesta pesquisa ajudou a guiar as escolhas nos momentos em que os resultados pareciam não fazer sentido. Todavia, por ser um processo recursivo, o retorno aos fundamentos dessa metodologia prevê que haja ajustes necessários, a fim de se atingir os objetivos da proposta.

A proposta metodológica delineada para esta pesquisa contemplou todas as fases do KDD, conforme descrito a seguir.

\vspace{5mm}

\subsubsection{Fases da Mineração ao KDD}

Seleção: Nesta etapa foram coletadas as informações provenientes das bases de dados da Policia Rodoviária Federal de Pernambuco entre 2007 e 2015. Segundo informações da própria PRF/PE, apenas a partir de 2007 esses dados passaram a ser armazenados eletronicamente. A PRF/PE dinponibiliza em formato de Banco de Dados relacionais, alguns desses dados na Internet.
Contudo, no artigo “Uma análise da qualidade dos dados relativos aos boletins de ocorrências das rodovias federais para o processo de Mineração de Dados”, COSTA \& BERNARDINI \cite{Costa2015} destacam a não padronização e não aceitação dos dados pela comunidade internacional. EAVES, D. \cite{Eaves} sugere que os dados sejam disponibilizados na maneira como foram
coletados. 

\vspace{5mm}

A primeira base de dados coletada diretamente dos servidores da PRF continha dados de acidentes rodoviários e a segunda base de dados das interdições rodoviárias. 
A partir dos dados capturados na base da PRF utilizamos como variáveis de entrada:

\begin{itemize}
 \item Condição da Pista: {Seca, Com buracos, Molhada, Em obras, Com material granulado, Oleosa, Enlameada, Com gelo, Outras}
 \item Restrição de visibilidade: {Inexistente, Veículo Estacionado, Poeira/Fumaça/neblina, Vegetação, Ofuscamento, Cartazes/faixas, Placas}
 \item Traçado da via: {Reta, Curva, Cruzamento, Defeito}
 \item Tipo de veículo: {Automóvel, Caminhonete, Motocicletas, Caminhão, Caminhão-trator, Bicicleta, Ônibus, Motoneta, Micro-ônibus, Trator de rodas, Carroça, Caminhão-Tanque, Semi-Reboque, Utilitário, Ciclomotor, Charrete, Carro-de-mão, Quadriciclo, Trator misto, Reboque, Trator de esteiras, Não informado, Não se aplica, Não identificado}
\end{itemize}

\vspace{5mm}

Preprocessamento: Nesta fase foram retiradas as variáveis que continham inconsistência e “missing data”, como, por exemplo, informações acerca de latitude e longitude. Cabe destacar que a base de dados, como um todo, apresentava sérias inconsistências, uma vez que, por exemplo,
um mesmo acidente, quando envolvia dois ou mais veículos,
era lançado na base duas ou mais vezes, em função da
quantidade de veículos envolvidos. Foram eliminadas variáveis
em duplicidade da variável ``Data'' (i.e. as variáveis Mês, Ano que apareciam
separadamente).

Transformação: Foram criadas as variáveis “Tipo de
paralisação”, contemplando acidentes sem mortos e com, no
máximo, dois veículos envolvidos; “Dias da semana”
(domingo, segunda-feira,....sábado); “Ajuste de horas” (i.e. 17h58, 17h59, 18h, 18h01, 18h02, arredondadas para 18h);
“Ajuste de Km” (seguiu a mesma lógica do ajuste de horas).

\vspace{5mm}

Mineração de dados: O algoritmo escolhido para essa fase da pesquisa
foi Árvore de Decisão, que possibilita uma interpretação
imediata e de fácil compreensão. Como ferramentas, foram
escolhidas o Knime \cite{Knime}, R \cite{R-cran} e Weka \cite{Weka}, com objetivo
de estabelecer uma comparação entre eles, cuja intenção era
produzir um classificador mais robusto. Nessa direção, a
técnica Ensamble de classificadores \cite{Bernardini} estabelece que a
combinação de um ou mais classificadores iguais, ou mais de
um classificador diferente, aumenta a precisão. 

Tanto na ferramenta Knime como Weka o algoritmo de árvore de decisão é chamado de J48,
uma vez que se trata da implementação Java do algoritmo
C4.5 no R. A biblioteca “rparty” implementa esse algoritmo.

Para escolha das variáveis de entrada foi calculada a correlação linear entre todas as variáveis. Entre as variáveis BR e
Delegacia (variável que agrega municípios) obteve-se correlação linear de 0,653. Entre Tipo de Acidente e Traçado via a
correlação foi baixa, apenas 0,14. Variáveis com correlação linear abaixo disso foram descartadas. 

Outra métrica para a escolha de variáveis de entrada foi a entropia, que é um elemento considerado importante pela literatura \cite{NorvigRussel2004}. Nesse caso, variáveis que seriam desconsideradas em virtude da baixa correlação linear, foram reconsideradas por conta da entropia.

\vspace{5mm}

Interpretação/Avaliação: Produção de árvores de decisão a
partir do estabelecimento de diferentes nós-raízes, definidos em
virtude da correlação linear e/ou da entropia.

\vspace{1,5mm}

%++++++++++++++++++++++++++++++++++++++++++++++++++++++++++++++++++++++++++++++++++++++++++++++


\subsection{Dados encontrados antes da Mineração}

Os dados revelaram que a grande maioria dos acidentes ocorre com pista seca, sem restrição de visibilidade. Nos gráficos das Figuras 4.2 à Figura 4.23 a cor vermelha remete aos trechos das BRs em que ocorrem mais acidentes. A cor azul diz respeito à menor frequência de acidentes.

É possível perceber que há determinados locais na rodovia onde se concentram os acidentes. 

O terceiro gráfico, tipo ‘boxplot’, apresentados na sequência, são de cada rodovia identificada em trâmite, eles exibem a concentração das ocorrências em torno da mediana dessa localidade (Km).
 
Especulou-se, a priori, que a variável “traçado da rodovia” ou que as condições climáticas poderiam ser de grande influência na ocorrência de acidentes, contudo mais adiante descobrimos outros condicionantes que influenciam mais fortemente esses acontecimentos. 

\pagebreak
%++++++++++++++++++++++++++++++++++++++++++++++++++++++++++++++++++++++++++++++++++++++++++++++

Os gráficos das Figuras 4.2(1), 4.2(2)  e 4.3 contêm dados da BR 101, uma das mais importantes para o nordeste brasileiro, uma vez que atravessa a maioria dos estados dessa região, nas localidades mais densamente povoadas, em virtude disso, seu tráfego intenso. 
A gráfico da Figura 4.2(1) representa os acidentes que ocorreram a cada hora (abcissa) em cada Km (ordenada) nos últimos nove anos. 
O  gráfico da Figura 4.2(2) corresponde à frequência do local onde ocorreram esses acidentes. 

É possível perceber no gráfico das Figuras 4.2(1) e no 4.3 que especialmente em determinados locais (Km) -- por exemplo na BR 101, entre os Km 40 e 100 -- os acidentes ocorrem desde as 05h da manhã até as 23h. 

\begin{figure}[h]
	\caption{BR 101: Hora do acidente (1) Concentração em torno da hora (2)}
	\includegraphics[width=7cm,height=7cm]{Figuras/Preprocess/br101.png}
	\includegraphics[width=7cm,height=7cm]{Figuras/Preprocess/br101_2.png}
	
\end{figure}

\quad \quad
\begin{figure}[h]
	\centering
	\caption{ Frequência}
	\includegraphics[width=7cm,height=7cm]{Figuras/Preprocess/br101_4.png}
\end{figure}

\pagebreak
%++++++++++++++++++++++++++++++++++++++++++++++++++++++++++++++++++++++++++++++++++++++++++++++

O gráfico das Figuras 4.4(1), 4.4(2) e 4.5 apresentam dados da BR 104, que atravessa seis municípios de Pernambuco, dentre eles Caruaru, que é responsável por uma das maiores frotas de veículos do interior e por onde passam cerca de 50 mil veículos por dia. 
A gráfico da Figura 4.4(1) representa, nos últimos nove anos, os acidentes que ocorreram a cada hora (abcissa) em cada Km (ordenada).
O  gráfico da Figura 4.4(2) (boxplot) apresenta as ocorrências em torno da mediana dessa localidade (Km).
O terceiro gráfico (4.5) corresponde à frequência do local onde ocorreram esses acidentes. 
Percebe-se que em torno do Km 60 concentra-se o maior número de ocorrências. 
No gráfico 1 são identificados padrões, por exemplo no Km 60 ocorrem acidentes que se estendem das 04h às 23h. 



\begin{figure}[h]
	\caption{BR 104: Hora do acidente (1) Concentração em torno da hora (2)}
	\includegraphics[width=7cm,height=7cm]{Figuras/Preprocess/br104_12.png}
	\includegraphics[width=7cm,height=7cm]{Figuras/Preprocess/br104_2.png}

\end{figure}

\quad \quad
\begin{figure}[h]
	\centering
	\caption{ Frequência}
	\includegraphics[width=7cm,height=7cm]{Figuras/Preprocess/br104_3.png}
\end{figure}

\pagebreak
%++++++++++++++++++++++++++++++++++++++++++++++++++++++++++++++++++++++++++++++++++++++++++++++

O gráfico das Figuras 4.6 (1) e 4.6 (2) representa a BR 110. Está é a menor rodovia federal de Pernambuco. Esta rodovia inicia-se e termina em Ibimirim. Esta rodovia também pode ser considerada como a que possui o menor número de sinistro, consequentemente a mais segura. O local com maior frequência de acidentes se encontra no entorno do quilômetro 110, seguido pelos quilômetros 130 a 150.

\begin{figure}[h]
	\caption{BR 110: Hora do acidente (1)  Concentração em torno da hora (2)}
	\includegraphics[width=7cm,height=7cm]{Figuras/Preprocess/br110_1.png}
	\includegraphics[width=7cm,height=7cm]{Figuras/Preprocess/br110_2.png}

\end{figure}


\quad \quad
\begin{figure}[h]
	\centering
	\caption{ Frequência}
	\includegraphics[width=7cm,height=7cm]{Figuras/Preprocess/br110_3.png}
\end{figure}


\pagebreak
%++++++++++++++++++++++++++++++++++++++++++++++++++++++++++++++++++++++++++++++++++++++++++++++

O gráfico das Figuras 4.8(1), 4.8(2)  e 4.9 apresentam dados da BR 116, que percorre os estados do Brasil que vão desde o Rio Grande do Sul até o Ceará. O gráfico da Figura 4.8(1) mostra os acidentes que ocorreram a cada hora (abcissa) em cada Km (ordenada), também nos últimos nove anos. 
O  gráfico da Figura 4.8(2) corresponde à frequência do local onde esses acidentes aconteceram. Há determinados locais na rodovia em que a maioria dos acidentes se concentram. 
Gráficos tipo ‘boxplot’ exibem a concentração de ocorrências em torno da mediana, no caso da localidade da ocorrência (o Km na rodovia). 
 
É possível perceber no gráfico da Figura 4.9, que os acidentes em torno do Km 30 ocorrem com mais frequência a partir da 04h da manhã, estendendo-se até próximo às 22h. 


\begin{figure}[h]
	\caption{BR: 116 Hora do acidente (1) Concentração em torno da hora (2)}
	\includegraphics[width=7cm,height=7cm]{Figuras/Preprocess/br116_1.png}
	\includegraphics[width=7cm,height=7cm]{Figuras/Preprocess/br116_2.png}

\end{figure}

\quad \quad
\begin{figure}[h]
	\centering
	\caption{ Frequência}
	\includegraphics[width=7cm,height=7cm]{Figuras/Preprocess/br116_3.png}
\end{figure}


\pagebreak
%++++++++++++++++++++++++++++++++++++++++++++++++++++++++++++++++++++++++++++++++++++++++++++++

O gráfico das Figuras 4.10(1), 4.10(2)  e 4.11 trazem dados da BR 232, cuja importância se destaca pelo fato de atravessar todo o estado de Pernambuco, de leste a oeste. O primeiro gráfico apresenta os acidentes dos últimos nove anos, em cada hora (abcissa) e Km (ordenada). O  gráfico da Figura 4.10(2) corresponde ao local onde aconteceram esses acidentes. É possível perceber no gráfico da Figura 4.11, que nessa BR há um número maior de acidentes nos Km 0, 90, 110, 260, 410 e 500, desde a 00h até as 23h. 
Especulou-se a priori que a variável “traçado da rodovia” ou que as condições climáticas eram as principais responsáveis pelo grande número de acidentes.

\begin{figure}[h]
	\caption{BR 232: Hora do acidente (1)  Concentração em torno da hora (2)}
	\includegraphics[width=7cm,height=7cm]{Figuras/Preprocess/br232_1.png}
	\includegraphics[width=7cm,height=7cm]{Figuras/Preprocess/br232_32.png}

\end{figure}

\quad \quad
\begin{figure}[h]
	\centering
	\caption{ Frequência}
	\includegraphics[width=7cm,height=7cm]{Figuras/Preprocess/br232_3.png}
\end{figure}




\pagebreak
%++++++++++++++++++++++++++++++++++++++++++++++++++++++++++++++++++++++++++++++++++++++++++++++

O gráfico das Figuras 4.12(1), 4.12(2)  e 4.13 trazem dados da BR 316. Esta BR é uma das menores do Estado de Pernambuco, de oeste a leste. O primeiro gráfico apresenta uma área em branco no centro, os acidentes se concentram nas extremidades, no entorno do km 90 e 300 (da abcissa) e acontecem com maior frequência em torno das 17h00 (ordenada).

\begin{figure}[h]
	\caption{BR 316: Hora do acidente (1) Concentração em torno da hora (2)}
	\includegraphics[width=7cm,height=7cm]{Figuras/Preprocess/br316_1.png}
	\includegraphics[width=7cm,height=7cm]{Figuras/Preprocess/br316_2.png}

\end{figure}

\quad \quad
\begin{figure}[h]
	\centering
	\caption{ Frequência}
	\includegraphics[width=7cm,height=7cm]{Figuras/Preprocess/br316_3.png}
\end{figure}



\pagebreak
%++++++++++++++++++++++++++++++++++++++++++++++++++++++++++++++++++++++++++++++++++++++++++++++

Na BR 407 os acidentes se concentram na altura do Km 130, representados nos gráficos das Figuras 4.14 (1) e 4.14 (2).
Esta BR situa-se no extremo oeste do Estado, ligando a cidade de Afrânio a Petrolina. A peculiaridade desse trecho (km 130) é verificada na descida da ponte que atravessa o rio São Francisco.

\begin{figure}[h]
	\caption{BR 407: Hora do acidente (1) Concentração em torno da hora (2)}
	\includegraphics[width=7cm,height=7cm]{Figuras/Preprocess/br407_1.png}
	\includegraphics[width=7cm,height=7cm]{Figuras/Preprocess/br407_3.png}

\end{figure}

\quad
\begin{figure}[h]
	\centering
	\caption{ Frequência}
	\includegraphics[width=7cm,height=7cm]{Figuras/Preprocess/br407_2.png}
\end{figure}



\pagebreak
%++++++++++++++++++++++++++++++++++++++++++++++++++++++++++++++++++++++++++++++++++++++++++++++

O gráfico das Figuras 4.16 (1), 4.16 (2)  e 4.17 representam a BR 408. Em Pernambuco esta BR liga a cidade de Timbaúba a Jaboatão dos Guararapes. Esta BR tem importância a medida integra o setor industrial de autopeças no norte de Pernambuco proporcionado pela fábrica da Fiat (FCA) ligando-se ao polo industrial do sul do Estado. 

\begin{figure}[h]
	\caption{BR 408: Hora do acidente (1) Concentração em torno da hora (2)}
	\includegraphics[width=7cm,height=7cm]{Figuras/Preprocess/br408_1.png}
	\includegraphics[width=7cm,height=7cm]{Figuras/Preprocess/br408_2.png}

\end{figure}

\quad \quad
\begin{figure}[h]
	\centering
	\caption{ Frequência}
	\includegraphics[width=7cm,height=7cm]{Figuras/Preprocess/br408_3.png}
\end{figure}


\pagebreak
%++++++++++++++++++++++++++++++++++++++++++++++++++++++++++++++++++++++++++++++++++++++++++++++

A BR 423, Figuras 4.18 (1), 4.18 (2) e 4.19, esta BR é a mais estratégica rodovia do Nordeste por ser o elo entre a BR 232 (a mais extensa BR do Estado de Pernambuco) à principal hidroelétrica da região; Paulo Afonso. Esta rodovia inicia-se no município de São Caetano indo até a fronteira com o estado de Alagoas no município de Itaíba, com 196,2 km é a 4ª mais extensão do Estado.

\begin{figure}[h]
	\caption{BR 423: Hora do acidente (1)  Concentração em torno da hora (2)}
	\includegraphics[width=7cm,height=7cm]{Figuras/Preprocess/br423_1.png}
	\includegraphics[width=7cm,height=7cm]{Figuras/Preprocess/br423_2.png}
	
\end{figure}

\quad \quad
\begin{figure}[h]
	\centering
	\caption{ Frequência}
	\includegraphics[width=5cm,height=6cm]{Figuras/Preprocess/br423_3.png}
\end{figure}


\pagebreak
%++++++++++++++++++++++++++++++++++++++++++++++++++++++++++++++++++++++++++++++++++++++++++++++

A rodovia BR 424 inicia em Arcoverde e termina em Correntes. Com extensão de 133,9 km tem características semelhantes à BR 423 (vide gráficos das Figuras 4.20 (1 e 2) e 4.21), contudo esta rodovia tem importância secundária interligando o importante município do Agreste, Garanhuns, à BR 424.

\begin{figure}[h]
	\caption{BR 424: Hora do acidente (1) Concentração em torno da hora (2)}
	\includegraphics[width=7cm,height=7cm]{Figuras/Preprocess/br424_1.png}
	\includegraphics[width=7cm,height=7cm]{Figuras/Preprocess/br424_2.png}
	
\end{figure}

\quad \quad
\begin{figure}[h]
	\centering
	\caption{ Frequência}
	\includegraphics[width=7cm,height=7cm]{Figuras/Preprocess/br424_3.png}
\end{figure}


\pagebreak
%++++++++++++++++++++++++++++++++++++++++++++++++++++++++++++++++++++++++++++++++++++++++++++++

A BR 428 é a 11ª rodovia federal estudada nesta pesquisa, ela conecta os municípios de Belém do São Francisco ao município de Petrolina, o maior do Sertão pernambucano. Esta rodovia contorna o Rio São Francisco pela margem norte. Nesta rodovia o ponto crítico situa-se no entorno do km 180 sendo que, cerca das 19h00 o horário que ocorrem o maior número de acidentes (vide Figura 4.22).  

\begin{figure}[h]
	\caption{BR 428: Hora do acidente (1) Concentração em torno da hora (2)}
	\includegraphics[width=7cm,height=7cm]{Figuras/Preprocess/br428_1.png}
	\includegraphics[width=7cm,height=7cm]{Figuras/Preprocess/br428_2.png}
	
\end{figure}

\quad \quad
\begin{figure}[h]
	\centering
	\caption{ Frequência}
	\includegraphics[width=7cm,height=7cm]{Figuras/Preprocess/br428_3.png}
\end{figure}


\pagebreak
%++++++++++++++++++++++++++++++++++++++++++++++++++++++++++++++++++++++++++++++++++++++++++++++

O número elevado de acidentes que tem por causa o automóvel de passeio que trafega por via retilínea, provavelmente condutores comuns, não profissionalizados.
O Caminhão é o segundo veículo que mais se envolve em acidentes, seguido das motonetas (motocicletas com potência limitada). 

\begin{figure}[!ht]
\begin{center}
\caption{Tipo de Veículo X Num. Acidentes}
\includegraphics[width=150mm, height=90mm]{Figuras/Preprocess/TipoVeiculoXNumAciden.png}\\
\tiny Fonte: o Autor
\end{center}
\end{figure}

%++++++++++++++++++++++++++++++++++++++++++++++++++++++++++++++++++++++++++++++++++++++++++++++
Os dados do gráfico da Figura 4.25 demonstram que o traçado da via (em linha reta) não parece influenciar a causa  elevada no número de acidentes, pois, a maioria dos acidentes ocorre em pista retilínea sugerindo assim que o condutor é o principal responsável pelas ocorrências nas BRs desse tipo de sinistro, com isso nosso foco se direciona para analisar e tentar antever o comportamento do condutor, perante às condições externa, para além das condições da rodovia.

\begin{figure}[ht]
\begin{center}
\caption{Traçado da via X Num. Acidentes}
\includegraphics[width=150mm, height=50mm]{Figuras/Preprocess/TracadoViaNumAcident.png}\\
\tiny Fonte: o Autor
\end{center}
\end{figure}




\pagebreak



\subsection{Dados encontrados após a Mineração}

Os resultados dos classificadores serão demonstrados a
seguir.

As variáveis “Tipo de Acidente”, “Gravidade” e
“BRajustada” foram escolhidas pelas características de ganho
de informação, dado pelo cálculo da entropia. A variável “BRajustada”
significa que essa variável foi transformada de dado numérico para categórico. Na literatura \cite{NorvigRussel2004} aconselha-se que o nó da raiz dos classificadores, em especial na Árvores de Decisão, seja aquele que apresentam maior
entropia, como encontrado na variável “Tipo de Acidente”.  
A seguir, apresenta-se a métrica para avaliar um classificador, também conhecida como acurácia.

\begin{itemize}
	\item TP: True Positive;
	\item FP: False Positive;
	\item Prec.: Precison = TP/(TP +FP);
	\item Recall = TP/ (TP + FN);
	\item F-Me: F-measure ou f-score = 2 * Precison * Recall / (Precision + Recall);
	\item AUC: Area Under Curve (Roc);
\end{itemize}
    
\subsection{Métrica dos classificadores}

\begin{enumerate}
	\item[(i)] Variável: Tipo de Acidente (Entropia: 3.0686)
		\begin{table}[!ht]
			\centering
			%\caption{Volume de dados no mundo}
			\vspace{1mm}
			\begin{tabular}{l|c|c}
				\hline
				\textbf{Descrição} & \textbf{Valores} & \textbf{Percentual} \\
				\hline
				Instâncias Corretamente Classificadas & 7987 & 47.6324\% \\
				Instâncias Incorretamente Classificadas & 8781 & 52.3676\% \\
				Erro médio absoluto & 0.0786 & ---  \\
				Erro médio quadratico & 0.2083 & --- \\
			\end{tabular}
			\\
			\tiny Fonte: o Autor
		\end{table}
		
	%	\pagebreak
		
		\begin{table}[!ht]
			\centering
			\caption{Detalhe da acurácia para classe Tipo Acidente}
			\vspace{1mm}
			\begin{tabular}{l|c|c|c|c|c|l}
				\hline
				\textbf{TP} & \textbf{FP} & \textbf{Prec.} & \textbf{Recall} & \textbf{F-Me.} & \textbf{AUC} & \textbf{Classe} \\
				\hline
				0.337 & 0.059 & 0.372 & 0.337 & 0.354 & 0.738 & Colisão transversal \\
				0.026 & 0.012 & 0.066 & 0.026 & 0.038 & 0.684 & Colisão com objeto fixo \\
				0.925 &	0.003 &	0.920 & 0.925 & 0.923 & 0.980 & Atropelamento de pessoa \\
				0.463 &	0.157 &	0.448 &	0.463 &	0.455 &	0.731 &	Colisão lateral \\
				0.682 &	0.259 & 0.545 & 0.682 & 0.606 & 0.773 & Colisão traseira \\
				0.485 & 0.024 & 0.409 & 0.485 & 0.443 & 0.893 & Queda de Moto/bicicleta \\
				0.322 & 0.002 & 0.528 & 0.322 & 0.400 & 0.744 & Colisão com bicicleta \\
				0.122 & 0.026 & 0.229 & 0.122 & 0.159 & 0.786 & Capotamento \\
				0.890 & 0.014 & 0.655 & 0.890 & 0.755 & 0.954 & Atropelamento de animal \\
				0.048 & 0.007 & 0.243 & 0.048 & 0.081 & 0.729 & Colisão frontal \\
				0.440 & 0.089 & 0.366 & 0.440 & 0.399 & 0.792 & Saída de Pista \\
				0.000 & 0.000 & 0.000 & 0.000 & 0.000 & 0.658 & Colisão c/ objeto móvel\\
				0.096 & 0.006 & 0.292 & 0.096 & 0.144 & 0.774 & Tombamento \\
				0.000 & 0.000 & 0.000 & 0.000 & 0.000 & 0.616 & Derramamento de Carga \\
				0.041 & 0.000 & 0.400 & 0.041 & 0.074 & 0.627 & Danos Eventuais \\
				0.000 & 0.000 & 0.000 & 0.000 & 0.000 & 0.733 & Incêndio \\	
			\end{tabular}
			\\
			\tiny Fonte: o Autor
		\end{table}
		
		\begin{table}[!ht]
			\centering
			\caption{Matriz de confusão para a variável Tipo de acidente}
			\vspace{1mm}
			\begin{tabular}{l|c|c|c|c|c|c|c|l}
				\hline
				\textbf{a} & \textbf{b} & \textbf{c} & \textbf{d} & \textbf{e} & \textbf{f} & \textbf{g} & \textbf{h} & \textbf{Classificadas}\\
				\hline
				527 & 7 & 2 & 385 & 483 & 46 & 2 & 24 & Colisão transversal \\
				16 & 14 & 0 & 69 & 154 & 15 & 0 & 47 & Colisão com objeto fixo \\
				8 & 0 & 483 & 16 & 14 & 0 & 0 & 0 & Atropelamento de pessoa \\
				336 & 30 & 8 & 1674 & 1217 & 102 & 8 & 48 & Colisão lateral \\
				250 & 51 & 9 & 835 & 3573 & 105 & 11 & 59 & Colisão traseira \\
				44 & 4 & 1 & 74 & 120 & 266 & 2 & 0 & Queda de Moto/bicicleta \\
				8 & 0 & 0 & 22 & 38 & 3 & 38 & 1 & Colisão com bicicleta \\
				28 & 34 & 5 & 85 & 236 & 1 & 2 & 120 & Capotamento \\
				-- & -- & -- & -- & -- & -- & -- & -- & -- \\	
			\end{tabular}
			\\
			\tiny Fonte: o Autor
		\end{table}
		
		Os valores restantes foram omitidos por não representarem uma amostra
		adequada, pois a acurácia foi consideravelmente baixa, por exemplo o classificado não acerta na maioria das vezes qual a classe deve ser escolhida para todas os atributos. As variáveis de classe são as mesmas da tabela
		anterior. \\
					
%	\pagebreak	
	\item[(ii)] Variável: Gravidade (Entropia: 0,9997)
	\begin{table}[!ht]
		\centering
		%\caption{Volume de dados no mundo}
		\vspace{1mm}
		\begin{tabular}{l|c|c}
			\hline
			\textbf{Descrição} & \textbf{Valores} & \textbf{Percentual} \\
			\hline
			Instâncias Corretamente Classificadas & 12110 & 72.2209\% \\
			Instâncias Incorretamente Classificadas & 4658 & 27.7791\% \\
			Erro médio absoluto & 0.3816 & ---  \\
			Erro médio quadratico & 0.4368 & --- \\
		\end{tabular}
		\\
		\tiny Fonte: o Autor
	\end{table}
	
%\pagebreak	
	
	\begin{table}[!ht]
		\centering
		\caption{Detalhe da acurácia para classe Gravidade}
		\vspace{1mm}
		\begin{tabular}{l|c|c|c|c|c|l}
			\hline
			\textbf{TP} & \textbf{FP} & \textbf{Prec.} & \textbf{Recall} & \textbf{F-Me.} & \textbf{AUC} & \textbf{Classe} \\
			\hline
			0.907 & 0.608 & 0.727 & 0.907 & 0.807 & 0.721 & S \\
			0.392 & 0.093 & 0.703 & 0.392 & 0.504 & 0.721 & N \\
				
		\end{tabular}
		\\
		\tiny Fonte: o Autor
	\end{table}
	
	\begin{table}[!ht]
		\centering
		\caption{Matriz de confusão para a variável Gravidade}
		\vspace{1mm}
		\begin{tabular}{l|c|l}
			\hline
			\textbf{a} & \textbf{b} & \textbf{Classificadas}\\
			\hline
			9747 & 996 & a = S \\
			3662 & 2363 & b = N \\
		\end{tabular}
		\\
		\tiny Fonte: o Autor
	\end{table}
	
\pagebreak
	
	\item[(iii)] Variável: BRajustada (Entropia: 2,4128)
	\begin{table}[!ht]
		\centering
		%\caption{Volume de dados no mundo}
		\vspace{1mm}
		\begin{tabular}{l|c|c}
			\hline
			\textbf{Descrição} & \textbf{Valores} & \textbf{Percentual} \\
			\hline
			Instâncias Corretamente Classificadas & 13507 & 80.5522\% \\
			Instâncias Incorretamente Classificadas & 3261 & 19.4478\% \\
			Erro médio absoluto & 0.0469 & ---  \\
			Erro médio quadratico & 0.1656 & --- \\
		\end{tabular}
		\\
		\tiny Fonte: o Autor
	\end{table}
	
	\begin{table}[!ht]
		\centering
		\caption{Detalhe da acurácia para classe BR}
		\vspace{1mm}
		\begin{tabular}{l|c|c|c|c|c|l}
			\hline
			\textbf{TP} & \textbf{FP} & \textbf{Prec.} & \textbf{Recall} & \textbf{F-Me.} & \textbf{AUC} & \textbf{Classe} \\
			\hline
			0.902 & 0.178 & 0.812 & 0.902 & 0.854 & 0.917 & BR101 \\
			0.873 & 0.003 & 0.957 & 0.873 & 0.913 & 0.992 & BR104 \\
			0.213 & 0.001 & 0.357 & 0.213 & 0.267 & 0.816 & BR110 \\
			0.457 & 0.003 & 0.669 & 0.457 & 0.543 & 0.961 & BR116 \\
			0.760 & 0.068 & 0.787 & 0.760 & 0.774 & 0.919 & BR232 \\
			0.893 & 0.006 & 0.800 & 0.893 & 0.844 & 0.985 & BR316 \\
			0.951 & 0.007 & 0.857 & 0.951 & 0.901 & 0.995 & BR428 \\
			0.761 & 0.012 & 0.693 & 0.761 & 0.725 & 0.974 & BR423 \\
			0.461 &	0.006 &	0.599 & 0.461 &	0.521 & 0.957 & BR424 \\
			0.814 & 0.001 & 0.961 & 0.814 & 0.881 & 0.999 & BR407 \\
			0.158 & 0.010 & 0.460 & 0.158 & 0.235 & 0.781 & BR408 \\
		\end{tabular}
		\\
		\tiny Fonte: o Autor
	\end{table}
	
	\begin{table}[!ht]
		\centering
		\caption{Matriz de confusão para a variável BRajustada}
		\vspace{1mm}
		\begin{tabular}{l|c|c|c|c|c|c|c|l}
			\hline
			\textbf{a} & \textbf{b} & \textbf{c} & \textbf{d} & \textbf{e} & \textbf{f} & \textbf{g} & \textbf{h} & \textbf{Classificadas}\\
			\hline
			6960 & 0 & 0 & 625 & 0 & 0 & 0 & 0 & a = BR101 \\
			0 & 1071 & 0 & 156 & 0 & 0 & 0 & 0  & b = BR104 \\
			0 & 0 & 0 & 625 & 0 & 0 & 26 & 11  & c = BR110 \\
			0 & 0 & 85 & 0 & 90 & 11 & 0 & 0  & d = BR116 \\
			970 & 9 & 0 & 3185 & 1 & 0 & 1 & 0  & e = BR232 \\
			0 & 0 & 27 & 11 & 377 & 7 & 0 & 0  & f = BR316 \\
			0 & 0 & 0 & 0 & 0 & 95 & 0 & 0  & g = BR407 \\
			643 & 0 & 0 & 66 & 0 & 0 & 0 & 0  & h = BR408 \\
			0 & 39 & 0 & 0 & 0 & 0 & 449 & 92  & i = BR423 \\
			0 & 0 & 0 & 625 & 0 & 0 & 172 & 154  & j = BR424 \\
			0 & 0 & 15 & 0 & 3 & 675 & 0 & 0  & k = BR428 \\			
		\end{tabular}
		\\
		\tiny Fonte: o Autor
	\end{table}	

\end{enumerate}


A área sob a curva ROC, AUC (Area Under Curve) mede a
relação de verdadeiros positivos contra os falsos positivos.
Quanto maior a área da curva tanto melhor será o
classificador. Portanto, um número de verdadeiros positivos
acima de 80\% (vide coluna AUC da Tabela 4.5), o número de falsos positivos próximo a 0\% (vide Tabela 4.5 coluna FP (Falsos Positivos))
traduzem uma área sob a curva ROC (AUC) que dá maior confiabilidade
aos testes.

\pagebreak

A variável “BRajustada” não teve o maior coeficiente de
entropia encontrado, contudo esta variável apresentou
índices de classificação das instâncias corretas acima dos 90\%, somente em dois casos ficou abaixo desse patamar que foram para a BR 110 (área = 0,816) e para a BR 408 (área = 0,781) e,
o menor índice de classificação incorreta dentre os três
classificadores utilizados, podemos constatar isso da seguinte forma, na primeira linha da Matriz de Confusão demonstrada na Tabela 4.6 temos:


\begin{table}[!ht]
	\centering
	\caption{Linhda da Matriz de confusão correspondente a BR 101}
	\vspace{1mm}
	\begin{tabular}{l|c|c|c|c|c|c|c|l}
		\hline
		\textbf{a} & \textbf{b} & \textbf{c} & \textbf{d} & \textbf{e} & \textbf{f} & \textbf{g} & \textbf{h} & \textbf{Classificadores}\\
		\hline
		6960 & 0 & 0 & 625 & 0 & 0 & 0 & 0 & BR101 \\
	\end{tabular}
	\\
	\tiny Fonte: o Autor
\end{table}

A Tabela 4.7 foi retirada da Tabela 4.6 correspondente à linha da BR 101. O classificador Árvore de Decisão encontrou 6960 (classe a) de instâncias corretamente e 625 (classe d) de instâncias erradas, as entradas restantes estão em zero, significa que o classificador não "errou" em outras classes. A área sob a curva ROC correspondente a BR 101 foi de 0,917 ou 91,7\%. \\

A segunda linha da Matriz de Confusão da Tabela 4.6 está demonstrada na Tabela 4.8.
 
\begin{table}[!ht]
	\centering
	\caption{Matriz de confusão para a variável BRajustada}
	\vspace{1mm}
	\begin{tabular}{l|c|c|c|c|c|c|c|l}
		\hline
		\textbf{a} & \textbf{b} & \textbf{c} & \textbf{d} & \textbf{e} & \textbf{f} & \textbf{g} & \textbf{h} & \textbf{Classificadores}\\
		\hline
		0 & 1071 & 0 & 156 & 0 & 0 & 0 & 0  & BR104 \\
	\end{tabular}
	\\
	\tiny Fonte: o Autor
\end{table} 
 
Como se pode constatar na Tabela 4.8 (BR 104) o classificador encontrou 1071 instâncias corretas (classe b)  e 156 erradas (classe d), com área sob a curva ROC de 0,992 ou 99,2\%.\\

A 4ª linha (de baixo para cima) da Tabela 4.5 correspondente à BR 408 é demonstrada na Tabela 4.9 

\begin{table}[!ht]
	\centering
	\caption{Matriz de confusão para a variável BRajustada}
	\vspace{1mm}
	\begin{tabular}{l|c|c|c|c|c|c|c|l}
		\hline
		\textbf{a} & \textbf{b} & \textbf{c} & \textbf{d} & \textbf{e} & \textbf{f} & \textbf{g} & \textbf{h} & \textbf{Classificadores}\\
		\hline
		643 & 0 & 0 & 66 & 0 & 0 & 0 & 0  & BR408 \\
	\end{tabular}
	\\
	\tiny Fonte: o Autor
\end{table} 

O classificador Árvore de Decisão encontrou a menor área sob a curva ROC para essa BR, (vide última linha da Tabela 4.5) na coluna AUC (AUC = 0,781) e o número de falsos positivos (FP) foi 0,10 (10\%), acreditamos que devido ao baixo número ocorrências (acidentes e paralisações) dessa rodovia o classificador obteve este desempenho, contudo considerado razoável \cite{ROC}.

Com os resultados encontrados e demonstrados acima pelo algoritmo Árvore de Decisão, a variável BRajustada foi escolhida para explicar o comportamento das rodovias. 
\\
Foram construídas Árvores de Decisão com as ferramentas R-cran, Weka e Knime. O Weka construiu uma árvore ininteligível, o R-cran será discutida a seguir e uma parte da Árvore de Decisão que o Knime construiu é demonstrada na Figura 4.26.


\begin{figure}[!ht]
\centering
\caption{Árvore de Decisão gerada pelo Knime}
\includegraphics[width=150mm, height=145mm]{Figuras/Metodologia/arvoreKnime.png}
\\
\tiny Ffonte: o Autor
\label{fig:arvoreKnime}
\end{figure}

A árvore construída pelo Knime para a mesma variável “Causa do Acidente” => velocidade incompatível.
Devido a grande quantidade de nós folha foi escolhida parte da Árvore onde foram classificadas os acidentes em dias-da-semana, com índices que chegaram a 100\%. O valor ``p-value'' abaixo de 0,05 é destacado na Figura 4.6, também para exemplificar, o nó folha classificou como causas dos acidentes: nas
quartas-feiras: “ultrapassagem indevida”; nas sextas-feiras:
“defeito na via”; e no sábado: “dormindo ao volante”.

\pagebreak

Contudo, os melhores resultados, de acordo com mais alta
precisão, segundo a métrica dos classificadores, foi a variável
“BRajustada” com nó raiz, com curva ROC acima dos 90\% em quase todas 
as classes. O classificador Naïve Bayes obteve um
desempenho semelhante, com essa
variável. Somente na BR 408 e BR 110 ficou abaixo, o que
confirma os valores encontrados pelo Weka.
Os valores das regras encontradas pelo algoritmo Árvore de Decisão com a variável “Delegacia”, (abaixo do nó raiz BRajustada) foram:

(a) “Delegacia” [1101(Região Metropolitana)], [BR 101],
[KM: 4], [Traçado da via: Reta], [Gravidade = S (acidente com
mortes) = 
[Causa Acidente: Falta atenção]
[Causa Acidente: Velocidade incompatível]
[Causa Acidente: Ultrapassagem indevida]
[Causa Acidente: Defeito mecânico]
[Causa Acidente: Não guardar distância]
[Causa Acidente: Dormindo]
[Causa Acidente: Ingestão de álcool]

(b) “Delegacia” [1101(Região Metropolitana)], [BR 232],
[KM: 17], [Condição pista: Seca], [Tipo Auto: automóvel]=
[Causa Acidente: Velocidade incompatível]
[Causa Acidente: Ultrapassagem indevida]
[Causa Acidente: Desobediência à sinalização]
[Causa Acidente: Não guardar distância]
[Causa Acidente: Dormindo]
[Causa Acidente: Ingestão de álcool]

Essa variedade de causas explica que o condutor dessa
região não respeita a sinalização, os limites de velocidade, dentre outros regras de trânsito. Pode-se dizer que é um condutor
indisciplinado, pois todas as causas de acidentes elencadas foram encontrados.
Caso se considere um raio de 50 Km no entorno da capital Recife, acredita-se que os motoristas têm a mesma
característica, pelo tipo de acidente que acomete nessa área.
Os valores das regras encontradas pelo algoritmo para a
variável “Tipo do Acidente” foram:
(a) “Tipo de Acidente” [região metropolitana]: [Atropelamento
de pessoa], [pista seca], [período: noite], [Br < 116 (101, 104)]
, [Dia da semana: terça-feira]:
[Gravidade = N (sem morte)], [Km <= 69] => falta de atenção.
[Gravidade = S (com morte)] => outras.

Tipo de Acidente: [Atropelamento de pessoa], [pista seca],
[período: noite], [Br < 116 (101, 104)] , [Dia da semana: sexta-
feira]:
[Gravidade = N (sem morte)], [Km <= 58] => falta de atenção.
[Gravidade = S (com morte)] => [Km > 58] [Km <= 67] =>
falta de atenção.

\vspace{5mm}

A falta de atenção foi condição ``sine qua non'' que determinou os acidentes na região metropolitana do Recife. Os dados revelam, ainda, que em torno do Km 67 encontra-se o maior número de acidentes com morte de todo estado de Pernambuco.

\vspace{5mm}

As Figura 4.27 e 4.28 correspondem ao mesmo ponto na BR 101, a Figura 4.27 foi obtida a partir da API do Google Maps, ela demonstra o local (aproximadamente o Km 70, Br 101 -- sul) destacado na Matriz de Mortos, Figura 4.28, logo abaixo. Ao ser consultado, este ponto nas Árvores de Decisão, para explicar as causas do alto índice de óbitos, constatou-se que eram, em sua maioria, mortes por atropelamento. 

A imagem do Google Maps, Figura 4.27, define que o local é próximo à CEASA, que é principal Centro de Abastecimento de alimentos da região, com um grande fluxo de pessoas -- muitas delas das comunidades do entorno -- e de veículos que vêm de diversas regiões do país, para comercialização dos produtos em grosso e varejo. Esse exemplo aponta para a ideia de extrapolação das ferramentas utilizadas nessa pesquisa.  

\pagebreak

%% BR 101

\begin{figure}[htbp!]
	\centering
	\caption{Km 70, BR 101 (Sul) Pernambuco}
	\label{fig:Km70BR101}
	\includegraphics[width=150mm, height=95mm]{Figuras/Resultados/Km70BR101}\\
	\tiny Fonte: Google Maps
\end{figure}

\vspace{7mm}

Em seguida a Matriz de Mortos e a Árvore de Decisão correspondente a esse trecho. Chamamos a atenção para o km 68 desta rodovia o número de óbitos somente neste local é o dobro dos outros. Provavelmente o local mais perigoso para se atravessar em todo Estado de Pernambuco.

\begin{figure}[htbp!]
	\centering
	\caption{Matriz de Mortos: Km 56 -- 78, BR 101 (Sul) Pernambuco}
	\label{fig:MatrizMortos2d-101}
	\includegraphics[width=120mm, height=90mm]{Figuras/Resultados/MM2d101}\\
	\tiny Fonte: o Autor
\end{figure}

\pagebreak


%% BR 104
Outro trecho, agora na BR 104 (rota da Sulanca) é demonstrado na Figura 4.29.
\begin{figure}[htbp!]
	\centering
	\caption{Km 64 e 67, BR 104 Pernambuco}
	\label{fig:Km70BR101}
	\includegraphics[width=150mm, height=95mm]{Figuras/Resultados/CaruaruRegiao}\\
	\tiny Fonte: Google Maps
\end{figure}

\vspace{7mm}

A Matriz de Mortos correspondente a esse trecho é a Figura 4.30, Destacamos os pontos: km 64 e km 67 desta rodovia o número de óbitos nestes locais é grande.

\begin{figure}[htbp!]
	\centering
	\caption{Matriz de Mortos: Km 64 e 67, BR 104 - Rota da Sulanca - Pernambuco}
	\label{fig:MatrizMortos2d-101}
	\includegraphics[width=120mm, height=90mm]{Figuras/Resultados/MM2d104}\\
	\tiny Fonte: o Autor
\end{figure}

\pagebreak


A região no entorno da BR 116, os acidentes com
mortes [Gravidade = S] ocorrem frequentemente às quinta-feira, envolvendo todos os tipos de veículos.
Os valores das regras encontradas pelo algoritmo para a
variável “Causa do Acidente” foram:
[Ingestão de álcool], [Tipo de auto: não identificado], [Período:
Manhã] o tipo de acidente => colisão traseira.
[Ingestão de álcool], [Tipo de auto: automóvel], [Traçado da
via: Reta], [Condição da pista: molhada], [Dia da semana]:
[Segunda-feira] => colisão frontal
[Terça-feira] => colisão transversal
[Quarta-feira] => colisão transversal
[Quinta-feira] => saída de pista
[Sexta-feira] => colisão traseira
[Sábado]: [BR = 232] => colisão traseira
	[BR > 232] => colisão frontal


\begin{figure}[htbp!]
	\centering
	\caption{Matriz de Gravidade 3D: Br 116 predição para quinta-feira}
	\label{fig:MatrizMortos2d-101}
	\includegraphics[width=120mm, height=100mm]{Figuras/Resultados/MGrav3D116}\\
	\tiny Fonte: o Autor
\end{figure}


\pagebreak

\subsection{Resultado dos outros classificadores utilizados nesta pesquisa}

Os dados gerados pelos classificadores -- Naïve Bayes e Redes Neurais no software Weka permitiram confirmar que os resultados encontrados pelas Árvores de Decisão são os mais adequados para o modelo proposto.

O resultado do algoritmo Naïve Bayes é demonstrado na Tabela 4.10 (Instâncias classificadas corretamente e incorretamente), pela Tabela 4.11 (Acurácia) e pela Tabela 4.12 (Matriz de Confusão).

	\begin{table}[!ht]
		\centering
		\caption{Instâncias classificadas e Erro médio}
		\vspace{1mm}
		\begin{tabular}{l|c|c}
			\hline
			\textbf{Descrição} & \textbf{Valores} & \textbf{Percentual} \\
			\hline
			Instâncias Corretamente Classificadas & 9232 & 73,3339\% \\
			Instâncias Incorretamente Classificadas & 3357 & 26,6661\% \\
			Erro médio absoluto & 0.0588 & ---  \\
			Erro médio quadrático & 0.1908 & --- \\
		\end{tabular}
		\\
		\tiny Fonte: o Autor
	\end{table}


	\begin{table}[!ht]
		\centering
		\caption{Detalhe da acurácia para classe BRajusta}
		\vspace{1mm}
		\begin{tabular}{l|c|c|c|c|c|l}
			\hline
			\textbf{TP} & \textbf{FP} & \textbf{Prec.} & \textbf{Recall} & \textbf{F-Me.} & \textbf{AUC} & \textbf{Classe} \\
			\hline
		0,947 &  0,277  & 0,734  &   0,947  & 0,827  & 0,914 & BR101\\
		0,895 &  0,008  & 0,893  &   0,895  & 0,894  & 0,997 & BR104\\
		0,159 &  0,000  & 0,538  &   0,159  & 0,246  & 0,996 & BR110\\
		0,833 &  0,010  & 0,510  &   0,833  & 0,633  & 0,995 & BR116\\
		0,429 &  0,052  & 0,748  &   0,429  & 0,546  & 0,834 & BR232\\
		0,589 &  0,016  & 0,510  &   0,589  & 0,546  & 0,983 & BR316\\
		0,772 &  0,003  & 0,901  &   0,772  & 0,832  & 0,996 & BR407\\
		0,165 &  0,013  & 0,284  &   0,165  & 0,209  & 0,890 & BR408\\
		0,607 &  0,013  & 0,650  &   0,607  & 0,628  & 0,983 & BR423\\
		0,377 &  0,006  & 0,594  &   0,377  & 0,461  & 0,988 & BR424\\
		0,875 &  0,009  & 0,849  &   0,875  & 0,862  & 0,994 & BR428\\
		
		\end{tabular}
		\\
		\tiny Fonte: o Autor
	\end{table}

O algoritmo Naïve Bayes encontrou o percentual de instâncias corretamente classificadas um pouco abaixo do percentual da   
Árvore de Decisão.
A área sob a curva ROC (AUC) para a variável BRajustada, o Naïve Bayes (NB) obteve valores próximos aos da Árvore de Decisão (AD) para quase todas as classes, com exceção para a área da classe BR 408:\\
-- para NB foi 0,890 \\ 
-- para AD foi 0,781 \\
diferença portanto de 10,9\% acima para o NB.

\pagebreak

A Tabela 4.12 corresponde à Matriz de Confusão do NB para a variável BRajustada.

	\begin{table}[!ht]
		\centering
		\caption{Matriz de confusão -- BRajustada -- Naïve Bayes}
		\vspace{1mm}
		\begin{tabular}{l|c|c|c|c|c|c|c|c|c|c|l}
			\hline
			\textbf{a} & \textbf{b} & \textbf{c} & \textbf{d} & \textbf{e} & \textbf{f} & \textbf{g} & \textbf{h} & 
			\textbf{i} & \textbf{j} & \textbf{k} & 		\textbf{Classificadas}\\
			\hline
			 5317 &  0  &  0 &  0 & 198 & 0  & 0 & 102 &   0  &  0  &  0 &   a = BR101\\
			 0 & 761 &  0 &  0 & 88 &  0  & 0 &   0 &   1  &  0  &  0 &   b = BR104\\
			 0 &  0  &  7 &  0 & 12 &  0  & 0 &   0 &  20  &  5  &  0 &   c = BR110\\
			 0 &   0 &  0 & 130 & 1 & 24  & 0 &   0 &   0  &  0  &  1 &   d = BR116\\
			 1605 &  69 &  0 &  0 & 1424 & 159 &   59  &  0   &  0  &  0 & 0 &  e = BR232\\
			 0 &   0 &  0 & 94 & 47 & 206 & 0 &   0 &   0  &  0  &  3 &   f = BR316\\
			 0 &   0 &  0 &  0 &  1 &  1 & 346 &  0 &   0  &  0 & 100 &   g = BR407\\
			 323 &   0 &  0 &  0 &  1 &  0 &  0 &  64 &   0  &  0  &  0 &   h = BR408\\
			 0 &  22 &  5 &  0 &  95 & 0 &  0 &   0 & 290  & 66  &  0 &   i = BR423\\
			 0 &   0 &  1 &  0 &  36 & 0 &  0 &   0 & 135  & 104 &  0 &   j = BR424\\
			 0 &   0 &  0 & 31 &  0  & 14 & 38 &  0 &   0  & 0 & 583 &    k = BR428\\			
		\end{tabular}
		\\
		\tiny Fonte: o Autor
	\end{table}	

As classes a, b, c,...,k correspondem à coluna ``Classificadas'' (BR 101, BR 104,...,BR 428). 
Podemos contatar que há um maior número de entradas diferentes de zero em cada linha dessa tabela, portanto o classificador NB "errou" mais vezes ao tentar encontrar determinadas classes corretas, por exemplo:

	\begin{table}[!ht]
		\centering
		\caption{Matriz de confusão -- BRajustada -- Naïve Bayes}
		\vspace{1mm}
		\begin{tabular}{l|c|c|c|c|c|c|c|c|c|c|l}
			\hline
			\textbf{a} & \textbf{b} & \textbf{c} & \textbf{d} & \textbf{e} & \textbf{f} & \textbf{g} & \textbf{h} & 
			\textbf{i} & \textbf{j} & \textbf{k} & 		\textbf{Classificadas}\\
			\hline
			5317 &  0  &  0 &  0 & 198 & 0  & 0 & 102 &   0  &  0  &  0 &   a = BR101			
		\end{tabular}
		\\
		\tiny Fonte: o Autor
	\end{table}	

Nas classes ``e'' e ``h'' da linha a = BR101, o classificador ``errou'', respectivamente em 198 e 102, portanto deveria ter entradas a zero caso tivesse classificado corretamente todas as instâncias.

	\begin{table}[!ht]
		\centering
		\caption{Matriz de confusão -- BRajustada -- Naïve Bayes}
		\vspace{1mm}
		\begin{tabular}{l|c|c|c|c|c|c|c|c|c|c|l}
			\hline
			\textbf{a} & \textbf{b} & \textbf{c} & \textbf{d} & \textbf{e} & \textbf{f} & \textbf{g} & \textbf{h} & 
			\textbf{i} & \textbf{j} & \textbf{k} & 		\textbf{Classificadas}\\
			\hline
			 1605 &  69 &  0 &  0 & 1424 & 159 &   59  &  0   &  0  &  0 & 0 &  e = BR232			
		\end{tabular}
		\\
		\tiny Fonte: o Autor
	\end{table}

Nas classes ``a'', ``b'', ``f'' e ``g'' da linha e = BR232, o classificador ``errou'', respectivamente, em 1605, 69, 159 e 59. Podemos constatar que na classe ``a'', o NB obteve mais erros (1605) do que acertos (1424), portanto ele não foi capaz de encontrar um quantidade adequada de classes corretas para e = BR232. Concluímos, assim, que o algoritmo Árvore de Decisão supera o Naïve Bayes.

\pagebreak

A Rede Neural utilizada na classificação dos pontos críticos das rodovias está descrita na Figura 4.32. As variáveis de entrada são descritas segundo a equação a seguir: 

\begin{equation}
ProbAcid = (RestVisibi * CondPista * TracadoVia * TipoAcident * CausaAcident) + Gravidade
\end{equation}
 
 

\begin{figure}[!ht]
	\centering
	\caption{Rede Neural utilizada na classificação}
	\includegraphics[width=110mm, height=85mm]{Figuras/Resultados/rnn101.png}\\
	\tiny Fonte: O Autor
	\label{fig:RNN2}
\end{figure}

As variáveis da equação 4.1 contemplam as variáveis da equação 3.1 mais as seguintes variáveis:

\begin{itemize}
	\item RestVisibi   - Restrição de Visibilidade;
	\item CondPista    - Condição da Pista;
	\item TraçadoVia   - Traçado da Via;
	\item TipoAcident  - Tipo de Acidente;
	\item CausaAcident - Causa do Acidente;
	\item Gravidade    - Gravidade do acidente.
\end{itemize}

A Equação 4.1 foi baseada nos parâmetros escolhidos para a Equação 3.1 que por sua vez foram baseados no entropia e correlação linear forte entre as variáveis. Ao acrescentar mais parâmetros à Equação 4.2 procurou-se chegar mais próximo dos resultados encontrados pelos outros classificadores utilizados nesta pesquisa, contudo os resultados foram irrelevantes. A Figura 4.33 demonstra os resultados da Rede Neural como classificador

A Figura 4.32 é um esboço da rede neural utilizada em um dos experimentos. Algumas variáveis como CausaAciden e TipoAcident não se encontram contempladas neste modelo, contudo nos experimentos elas não alteraram o comportamento do classificador.

A partir desse resultado podemos concluir que a Árvore de Decisão encontrou resultados mais promissores, apesar da taxa de erro médio absoluto (vide Figura 4.33) da Rede Neural ser o menor dos classificadores utilizados nesta pesquisa.

\begin{figure}[!ht]
	\centering
	\caption{Resultado da classificação feita pela Rede Neural -- acurácia}
	\includegraphics[width=90mm, height=115mm]{Figuras/Resultados/RNN}\\
	\tiny Fonte: O Autor
	\label{fig:RNN}
\end{figure}


\pagebreak

\section{Acoplamento com a estrutura dinâmica}

As predições feitas na primeira fase têm como ``output'' o georreferenciamento que localiza um ponto no mapa a partir do quilômetro (Km). O georreferenciamento mais preciso seria a latitude e longitude. Todavia, esses dados apresentaram grave inconsistência na base de dados da PFR/PE, tendo sido descartados. Sentimos necessidade de informações sobre a teoria do fluxo de tráfego, que consiste de leis matemáticas, físicas aplicadas ao tráfego de automóvel, tais como o fluxo (veículos/hora), a velocidade (km/hora) e a densidade de tráfego (veículos/km).

A estrutura dinâmica é composta por duas API's, uma disponibilizada pelo Google, através do Google Maps, que está atualmente na versão V3, e a
outra por uma API do Twitter. A API do Google Maps proporciona uma ``leitura'' atualizada e precisa, de forma que os dados do Km da rodovia podem ser localizados no mapa.

A API do Twitter, por sua vez, possibilita atualizar o utilizador com informações recentes. Contudo, o objetivo desta API é fazer 
um Arco Cibernético das informações, retroalimentando, com dados recentes, um banco de dados de redes sociais. Isso permite um visualização 
instantânea do ambiente como um todo.

\pagebreak

\subsection{Mineração em texto no Twitter}

A mineração de dados textuais na rede social Twitter demonstrou ser uma ferramenta promissora, uma vez que oferece uma ampla gama de informações, atualizadas em tempo real. Entretanto, para o caso de pesquisas dessa natureza, o monitoramento precisa ser constante, tendo em vista que novas informações são produzidas a todo momento; e outros canais precisam ser monitorados, a fim de ampliar o universo de dados disponíveis, para produzir o efeito esperado no modelo.

Para localizar novos canais de informação, foram construídos subgrafos contemplando, além dos tweets da PFR, retweets dos seguidores desse canal. Esperava-se, com isso, encontrar novos Hubs na rede. No entanto, a busca em subgrafos é uma tecnologia que precisa ser investigada mais a fundo, dada a sua complexidade.

Conceitos como intermediação, centralidade, peso das arestas, precisam ser adequadamente compreendidos pelo pesquisador, para que a tecnologia seja explorada em todo seu potencial. Isso implica no estudo de novos algoritmos, próprios para mineração em grafos de redes sociais, o que, em princípio, fugia do escopo dessa pesquisa.

Ainda que consideremos que a ferramenta não tenha sido adequadamente explorada, por todas as questões propostas acima, a mineração de dados textuais do twitter permitiu inferir que o utilizador dessa rede social faz referência ao que acontece nas rodovias, com especial atenção para fatos que possam ter implicação em seu cotidiano, tanto imediatamente (por exemplo, congestionamento numa via que ele utiliza para ir ao trabalho), quanto num universo temporal mais distante (por exemplo, condição da rodovia que dá acesso à cidade em que ele vai passar um feriado). Esse último aspecto é aquele que interessa particularmente ao modelo proposto nessa pesquisa.

%\vspace{5mm} -- e bigramas, termos com duas palavras

Os dados da Figura 3.34 mostram a busca dos termos frequentes encontrados no documento textual, extraído a partir de 3.200 tweets. O primeiro gráfico apresenta unigramas -- termos frequentes que contém uma palavra. Foram encontradas palavras como: fiscalização, colisão, vítima fatal, acidentes, detre outras. 


%\pagebreak

\begin{figure}[!ht]
\centering
\caption{Gráfico de frequência de palavras -- unigramas}
\includegraphics[width=0.7\linewidth]{Figuras/Twitter/freqPalavr}
\label{fig:freqPalavras}
\end{figure}

\qquad

A nuvem de palavras da Figura 4.35 é um gráfico de frequência de palavras em que os termos mais frequentes aparecem em destaque -- com letras em tamanho maior -- seguidos pelos próximos termos mais frequentes -- em tamanho um pouco menor -- e assim sucessivamente, chegando a contemplar dezenas ou até centenas de palavras, a depender da escolha do pesquisador. No caso da nuvem de palavras a seguir, aparecem cerca de 50 termos, e destacam-se como mais frequentes (em virtude do tamanho): veículo, BR, acidente, balanço, dentre outras. 

\begin{figure}[!ht]
\centering
\caption{Gráfico tipo Nuvem de Palavras referente aos unigramas}
\includegraphics[width=0.6\linewidth]{Figuras/Twitter/nuvem2.png}
\label{fig:Nuvem1}
\end{figure}

%\vspace{5mm}



O dendograma da Figura 4.36 é um gráfico que agrupa palavras de acordo com o assunto. 
No dendograma a seguir foram identificados seis agrupamentos (clusters), como, por exemplo: ferido, balanço e morto; envolvidos, veículo (veic), fiscalização (fisc), multa, alcoolemia, pessoa, test; PRF; fatal, vítima, envolvido, óbito, prisão, condutor, colisão, moto, ano, h (hora). Os agrupamentos, quando analisados adequadamente, permitem identificar, com um número reduzido de palavras, o acontecimento ao qual está sendo feita referência. Por exemplo, no último agrupamento exemplificado acima é possível deduzir que houve uma colisão envolvendo uma moto, que resultou no óbito de um dos envolvidos e na prisão daquele que foi o responsável pela colisão.   


\begin{figure}[!ht]
\centering
\caption{Dendograma de Agrupamento (clustering) do resultado da mineração}
\includegraphics[width=0.7\linewidth]{Figuras/Twitter//Cluster}
\label{fig:Cluster}
\end{figure}

\pagebreak
\vspace{15mm}


O Gráfico 4.37 ilustra a utilização de outro algoritmo de agrupamento, conhecido com K-means. 

\begin{figure}[!ht]
	\centering
	\caption{Gráfico da Agrupamento (clustering) do resultado a mineração}
	\includegraphics[width=0.9\linewidth]{Figuras/Twitter/Cluster2}
	\label{fig:Cluster2}
\end{figure}


Uma análise mais minuciosa dos gráficos produzidos pela mineração em textos pode trazer contribuições de considerável relevância, que permitem refinar o modelo de predição da Etapa 1, descrita na Figura 4.1 do modelo proposto.

%\section{Casos de teste do modelo preditivo}