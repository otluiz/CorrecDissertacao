\chapter{Conclusão}\label{intro:resumen}

Após a realização do estudo, os dados sugerem que o modelo de predição ora proposto dá conta das questões dessa pesquisa, ou seja, possibilita a gestão logística relativa a horários de utilização das rodovias.  Os resultados do apontam para a eficácia da aplicação da
I.A. para analisar questões relativas ao tráfego de veículos e
problemas que atingem as rodovias, comprometendo o
deslocamento daqueles que a utilizam, tanto para uso privado,
quanto relacionados ao contexto profissional e logístico. Nesse
sentido, órgãos federais de controle de estradas podem se
beneficiar de estudos dessa natureza. 
Os dados encontrados dessa pesquisa corroboram com os resultados encontrados em estudos semelhantes, seja no Brasil ou em outros países, sugerindo que há um padrão de
comportamento em rodovias que pode ser analisado, de
maneira a facilitar o transporte de cargas e tráfego de veículos em geral, em seu curso. 
A pesquisa também contribuiu para a compreensão das causas de constrangimentos e acidentes nas vias. Os dados revelaram que a maioria dos acidentes ocorre em via reta, com pista seca e em boas condições, sugerindo como uma das principais causas de acidente, a falta de atenção do condutor, que resulta, na maioria das vezes, em colisões traseiras e/ou laterais.
Os acidentes acontecem nos horários em que há mais veículos trafegando na via, independentemente da hora do dia. Quando a via exige maior atenção, por condições que lhes são peculiares (um cruzamento, por exemplo), uma pequena restrição para ser amplificada, aumentando consideravelmente a quantidade de acidentes.
Outra contribuição da pesquisa a ser destacada é de
cunho metodológico-prático. Do ponto de vista metodológico, pela contribuição da aplicação do processo CRISP-DM,
usado para construir o modelo preditivo. 
O algoritmo Árvores de Decisão mostrou-se robusto, quando aplicado a esse tipo de problema.
Quanto á mineração de textos em redes sociais, no caso do Twitter, possibilita a identificação de comportamentos do condutor, antes mesmo de utilizar a via. Todavia, embora tenha sido uma ferramenta útil, para que pudesse haver uma influência maior nos resultados, seria necessário ampliar o escopo para outras redes sociais que tenham o perfil de troca de informações pontuais sobre comportamento de usuários de rodovias.
Do ponto de vista prático, a contribuição da pesquisa se dá pela proposição de um modelo que integre predição à
API de mapas de posicionamento global, fornecendo
informação suficiente a um gestor para decidir quando enviar,
por exemplo, uma frota de caminhões por determinada rodovia
que apresente retenções crescentes de logística de cargas.
Tal modelo se configura como um avanço em relação às funcionalidades das soluções disponíveis que existem até o momento, tais como: Google Maps, Waze e outros dessa natureza somente exibem informações
momentâneas, produzidas e compartilhadas pelos utilizadores
dos aplicativos ou por informações provindas de GPS, com a
nossa abordagem dados históricos de rodovias podem auxiliar
predições sobre seu comportamento.
Pode-se destacar alguns aspectos que figuraram como elementos dificultadores na realização da pesquisa. As informações da base de dados da PRF ........................................................................................................................................

\section{Trabalhos futuros}

Essa pesquisa não encerra a questão proposta, com respeito ao desenvolvimento de um modelo preditivo. O que foi apresentado, sobretudo, foi a intenção de um modelo que servirá como ponto de partida para o desenvolvimento de uma ferramenta que atenda ao fim proposto, de forma eficaz. Nesse sentido, entendemos novas pesquisas precisam ser condizidas, para a ampliação do modelo sugerido. 
Trabalhos futuros incluem a incorporação desta proposta
em modelos formais de decisão, por exemplo de roteamento
rodoviário metropolitanos. A API Google Maps, o ``front-end'' do sistema, em uma futura aplicação poderá ser executada em um aparelho 
celular do tipo ``Smartphone'', com capacidade para executar aplicativos gráficos mais complexos.