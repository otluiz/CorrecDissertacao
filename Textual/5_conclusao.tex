\chapter{Conclusão}\label{intro:resumen}

As rodovias federais brasileiras que cruzam as regiões metropolitanas das grandes cidades se apresentarão sempre como um gargalo no 
fluxo do trasporte de cargas devido ao crescimento do número de veículos que nelas trafegam.

A solução proposta visa reduzir e dirimir os atuais gargalos burocráticos e tecnológicos
para obtenção das bases de dados históricas de entidades públicas, os direitos autorais 
para utilização de APIs e de tecnologias específicas necessárias para apropriação via Internet. 
Entendemos ser normal esse cuidado por se tratar de informações de órgão que devem primar 
pelas informações de seus usuários. Mas tentaremos suprir demandas reais e importantes 
sem que isso represente algum risco à privacidade dos geradores de dados. 
Em suma, nossa solução pretende mitigar o gargalo da logística de transporte de cargas, ofe-
recendo uma solução possível à gestão de frotas de veículos que trafegam em rodovias, 
notadamente no caso do entrono metropolitano do Recife.


\section{Discussão}

Algumas propostas para a RMR vêm sendo amplamente difundidas pelas mídias, tais como o arco metropolitano. Arcos metropolitanos, 
para além dos transtornos de se contruir um, são muito caros, requerem constantes manutenções  e com o passar dos anos, 
com o crescimento populacional no seu entorno, tornam-se novamente  um novo gargalo para o transporte de cargas.

Gerir como as rodovias são utilizadas é a maneira mais racional, elas estão ai para auxiliar no trasporte de pessoas, mercadorias e para serviços, portanto é de todos e todos têm o dever de contribuir preservando-as e respeitando o direito dos outros.


\section{Trabalhos futuros}


A API Google Maps, o ``front-end'' do sistema, em uma futura aplicação poderá ser executada em um aparelho 
celular do tipo ``Smartphone'', com capacidade para executar aplicativos gráficos mais complexos.