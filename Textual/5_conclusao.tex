\chapter{Considerações finais}\label{intro:resumen}

Após a realização dessa pesquisa, os dados sugerem que é possível propor um modelo de predição que possibilite a gestão logística relativa a dias e horários de utilização das rodovias.  

Os resultados apontam para a eficácia da aplicação da
I.A. para analisar questões relativas ao tráfego de veículos e problemas que atingem as rodovias, comprometendo o deslocamento daqueles que a utilizam, tanto para uso privado, quanto relacionados ao contexto profissional e logístico. 

Nesse sentido, órgãos federais de controle de estradas podem se beneficiar de pesquisas dessa natureza. 
Os dados encontrados nessa pesquisa corroboram com os resultados encontrados em estudos semelhantes, quer seja no Brasil ou em outros países, sugerindo que há um padrão de comportamento em rodovias que pode ser analisado, de maneira a facilitar o transporte de cargas e o tráfego de veículos em geral, em seu curso. 

A pesquisa também contribuiu para a compreensão das causas de constrangimentos e acidentes nas vias. Os dados revelaram que a maioria dos acidentes ocorre em via reta, com pista seca e em boas condições, sugerindo como uma das principais causas de acidente a falta de prudência do condutor, que resulta, na maioria das vezes, em colisões traseiras e/ou laterais, essa última sendo a que mais culmina em óbito. Numa análise restrita a acidentes com morte por atropelamento, os dados revelaram que próximo aos perímetros urbanos não há uma estrutura segura para o pedestre atravessar a via. Por exemplo, falta de sinalização, passarelas, limitador de velocidade e faixa de pedestre são fortes condicionantes para as situações de atropelamento. A quantidade de mortes por atropelamento entre os quilômetros 66 e 71 da BR 101 - mais de 9.500 mortes em nove anos - corroboram com essa análise.

Os acidentes acontecem nos horários em que há mais veículos trafegando na via, entre seis e nove horas da manhã, e entre quatro e sete horas da tarde. Quando a via exige maior atenção, por condições que lhes são peculiares (um cruzamento, por exemplo), uma pequena restrição pode ser amplificada, aumentando consideravelmente a quantidade de acidentes.

Outra contribuição da pesquisa a ser destacada é de cunho metodológico-prático. Destaca-se, inicialmente, a articulação entre os resultados envolvendo diferentes algoritmos. A pesquisa utilizou Naïve Bayes, TF-IDF para os dados minerados do Twitter; 
Árvores de Decisão, Redes Neurais e  Naïve Bayes, para o trabalho com os dados da PRF no modelo de classificação.  

Ainda do ponto de vista metodológico, ressalta-se a contribuição da aplicação do processo CRISP-DM,
utilizado para construir o modelo de mineração de dados. 
O algoritmo Árvores de Decisão mostrou-se robusto quando aplicado a esse tipo de problema.
Quanto à mineração de textos em redes sociais (Twitter), embora tenha sido uma ferramenta útil, há a necessidade de um monitoramento intensivo das redes sociais, para que pudesse haver uma influência maior nos resultados. Seria também necessário ampliar o escopo para outras redes sociais que pudessem informar sobre o comportamento de usuários de rodovias.

Do ponto de vista prático, a contribuição da pesquisa se dá pela proposição de um modelo que integre predição à
API de mapas de posicionamento global, fornecendo informação suficiente a um gestor para decidir quando enviar,
por exemplo, uma frota de caminhões por determinada rodovia que apresente retenções frequentes.

Em relação a outras pesquisas dessa natureza, algumas delas consideradas no Estado da Arte dessa dissertação, essa pesquisa avança em relação ao que até então foi proposto, pelo fato de que além de identificar ocorrências nas rodovias, levando em conta passado e presente, propõe um modelo que contempla o futuro, possibilitando ao usuário escolhas mais assertivas. Outro avanço em relação às pesquisas identificadas diz respeito à utilização de uma ampla gama de técnicas de I.A. para encontrar soluções, bem como, propor articulações entre essas técnicas. 

Alguns aspectos merecem destaque em relação às dificuldades enfrentada na proposição do modelo de classificação e predição. Em primeiro lugar, as informações da base de dados da PRF apresentavam muitas lacunas (missing values), devido ao tipo de registro feito na ocorrência. Por exemplo, um acidente que envolva dois veículos (ou mais) -- um motocicleta e um carro -- aparecia, muitas vezes, com dois registros: acidente envolvendo carro e acidente envolvendo moto. Como essa, várias situações de dados duplicados foram identificados, tornando mais complexo o trabalho de preprocessamento dos dados.

Ainda em relação ao registro dos dados pela PRF, dados relativos à latitude e longitude estavam ausentes ou preenchidos, muitas vezes, de forma equivocada, o que comprometeu a localização exata do evento (colisão, atropelamento), fazendo com que fosse necessário desprezar essas variáveis e encontrar uma alternativa, uma vez que a localização era fundamental para o georreferenciamento.

Foi percebida também certa imprecisão em relação à quilometragem em que se dava a ocorrência, com erros que variavam de alguns metros até quilômetros, sendo necessário o olhar atento do pesquisador e o confronto cuidadoso de informações, de modo que essas falhas fossem dirimidas.

Outra questão que merece destaque é o complexo processo de limpeza de dados da ``timeline'' do Twitter na etapa de preprocessamento. Atualmente, a própria PRF, para dar mais destaque, tem utilizado imagens para informar as ocorrências nas BRs, em vez de textos, o que tem contribuído para diminuir a quantidade de dados textuais. Com isso foi necessário procurar outras ``timelines'' para ampliar a quantidade de informações do Twitter.

\pagebreak

\section{Trabalhos futuros}

Essa pesquisa não encerra a questão proposta com respeito ao desenvolvimento de um modelo preditivo. O que foi apresentado, sobretudo, foi a intenção de um modelo que servirá como ponto de partida para o desenvolvimento de uma ferramenta que atenda ao fim proposto, de forma eficaz. Nesse sentido, entendemos que novas pesquisas precisam ser condizidas, para a ampliação do modelo sugerido. 

Trabalhos futuros incluem a incorporação desta proposta em modelos formais de decisão, por exemplo, de roteamento
rodoviário metropolitano. Através da API do twitter, implementar algoritmos de busca em redes sociais, para encontrar os Hubs difusores de informações.

A API Google Maps, o ``front-end'' do modelo proposto, em uma futura aplicação poderá ser executada em um aparelho 
celular do tipo ``Smartphone'', com capacidade para executar aplicativos gráficos mais complexos.
