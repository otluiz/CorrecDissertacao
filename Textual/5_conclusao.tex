\chapter{Considerações finais}\label{intro:resumen}

Após a realização desta pesquisa, os dados sugerem que é possível propor um modelo de predição que possibilite a gestão logística relativa a dias e horários de utilização das rodovias para os mais diversos atores que dela fazem uso.  

Os resultados encontrados apontam na mesma direção de trabalhos realizados no Brasil e em outros países  \cite{dos2016previsao} \cite{chong2004traffic} \cite{akgungor2009artificial} \cite{jadaan2014prediction} , que destacam para a eficácia da aplicação das técnicas de Iinteligência Artificial para analisar questões relativas ao tráfego de veículos e problemas que atingem as rodovias, comprometendo o deslocamento daqueles que a utilizam, tanto para uso privado, quanto relacionados ao contexto profissional e logístico e propor soluções baseadas em resultados científicamente testados. 

Os dados encontrados sugerem que há um padrão de comportamento em rodovias que pode ser analisado, de maneira a facilitar o tráfego de veículos em geral, em seu curso. 
Órgãos federais de controle podem se beneficiar de pesquisas desta natureza para compreender o comportamento do tráfego nas rodovias e propor soluções para os problemas que afetam esse cenário diariamente. 

Outro aspecto que pode impactar positivamente no georreferenciamento é o registro adequado e utilização constante, pelos órgãos de trânsito, das informações de latitude e longitude, que apareceram precariamente nos dados da PRF/PE. Essa medida facilitaria o trabalho de pesquisadores e profissionais que desenvolvem sistemas de posicionamento global. Por exemplo, o KM da BR 101 existe em Pernambuco, no Rio de Janeiro e, possivelmente, em outros locais. Assim, uma das principais informações utilizadas pela PRF para registro de ocorrências apresenta inconsistência.     

A pesquisa também contribuiu para a compreensão das causas de constrangimentos e incidentes nas vias. Os dados revelaram que a maioria dos acidentes ocorre em via reta, com pista seca e em boas condições, sugerindo como uma das principais causas de acidente a falta de prudência do condutor, que resulta, na maioria das vezes, em colisões traseiras e/ou laterais, essa última sendo a que mais culmina em óbito. 

Numa análise restrita a acidentes com morte por atropelamento, os dados revelaram que próximo aos perímetros urbanos não há uma estrutura segura para o pedestre atravessar a via, ou que o induza a obedecer as regras de trânsito, como por exemplo: atravessar somente na faixa de pedestre, quando estiver em trechos urbanos de rodovias; fazer uso dos semáforos para atravessá-las, quando houver; utilizar as passarelas (quando há) para cruzar as BRs com fluxo intenso. Outros condicionantes são, por exemplo: falta de sinalização, passarelas, limitador de velocidade e faixa de pedestre. A quantidade de mortes por atropelamento entre os quilômetros 66 e 71 da BR 101 - mais de 9.500 mortes em nove anos - apontam nessa direção.

Esse padrão de comportamento também foi identificado em outras pesquisas discutidas nesta dissertação e sugere que as variáveis do condutor são aquelas mais fortemente relacionadas a acidentes em rodovias, quer seja no perímetro urbano ou no interior. Isso pode indicar que não apenas no Brasil, mas em outros países, inclusive os mais desenvolvidos (como Estados Unidos, Portugal, Espanha e Itália) o motorista não obedece ás leis de trânsito, e mesmo com punições rigorosas e multas com altos valores, continuam dirigindo após ingestão de álcool, sem utilizar cinto de segurança, com velocidade acima da permitida, fazendo ultrapassagens indevidas, como revelam as pesquisas.
 
Nessa dissertação, identificamos que os acidentes acontecem nos horários em que há mais veículos trafegando na via, entre seis e nove horas da manhã, e entre quatro e sete horas da tarde. Quando a via exige maior atenção, por condições que lhes são peculiares (um cruzamento, por exemplo), uma pequena restrição pode ser amplificada, aumentando consideravelmente a quantidade de acidentes.

Outra contribuição da pesquisa a ser destacada é de cunho metodológico-prático. Destaca-se, inicialmente, a articulação entre os resultados envolvendo diferentes algoritmos. A pesquisa utilizou Naïve Bayes, TF-IDF para os dados minerados do Twitter; 
Árvores de Decisão, Redes Neurais e  Naïve Bayes, para o trabalho com os dados da PRF no modelo de classificação.

Observa-se que essa parece ser uma tendência nos trabalhos que envolvem análise de tráfego em rodovias \cite{beshah2010mining} \cite{olutayo2014traffic} \cite{thianniwet2010classification} \cite{de2015comparison} \cite{Bernardini}: a utilização de diferentes técnicas de I.A. para explicar o comportamento das rodovias. Todavia, o objetivo de parte dessas pesquisas foi a comparação entre as técnicas como a utilização de AD, RNA e NB com os resultados encontrdos em outras pesquisas, para estabelecer qual a melhor delas. No caso específico dessa pesquisa, o nosso interesse foi o de articular os resultados, não apenas identificando qual o melhor algoritmo, mas propor uma correlação entre eles.

Um exemplo que podemos dar a esse respeito, discutido no capítulo dedicado aos resultados, diz respeito ao fato de que a Matriz de Mortos e a Matriz de Gravidade nos deram informação sobre locais com maior número de acidentes, inclusive com óbitos, e no cruzamento com a árvore de decisão produzida identificamos a causa desses óbitos: morte por atropelamento. Ainda na perspectiva de articulação entre os dados, identificando a partir da API do Google Maps, que tratava-se de um lugar de grande tráfego de veículos e de pessoas, uma vez que era o trecho próximo à CEASA/PE. Assim, reiterando o que dissemos anteriormente, nossa pesquisa propõe, mais do que a comparação e o cruzamento entre os dados, uma análise muldimensional e pluritecnológica, ou seja, considerando mais de uma dimensão e levando em conta mais de uma tecnologia. 

Ainda do ponto de vista metodológico, ressalta-se a contribuição da aplicação do processo CRISP-DM,
utilizado para construir o modelo de mineração de dados. 
O algoritmo Árvores de Decisão mostrou-se robusto quando aplicado a esse tipo de problema.

Quanto à mineração de textos em redes sociais (Twitter), foi possível encontrar informações, classificá-las e extrair delas dados relevantes. Entretanto, um contraponto que é importante estabelecer refere-se ao fato de que não conseguimos estabelecer um valor numérico relativo a essas informações, para haver uma extrapolação para a equação do momentum, proposta para o estabelcimento do modelo de predição. 

Outro aspecto a ser mencionado é que embora tenha sido uma ferramenta útil, há a necessidade de um monitoramento intensivo das redes sociais, para que possa haver uma influência maior nos resultados. Seria também necessário ampliar o escopo para outras redes sociais que pudessem informar sobre o comportamento de usuários de rodovias, uma vez que nem sempre o utilizador dessa rede social está disposto a informar o que ele pretende fazer - como usuário de rodovias - por essa rede, cujas informações precisam ser sucintas, dadas as características do aplicativo. Ele pode, por exemplo, não propagar a informação no Twitter, mas fazê-lo no facebook.

Todavia, embora possamos admitir que o estudo das redes sociais precisaria ser mais aprofundado, destacamos que não foram encontrados estudos envolvendo o campo de nosso interesse - comportamento de tráfego em rodovias – e as redes sociais, com vistas à articulação entre mineração de dados e mineração de textos.  

Podemos ainda destacar a contribuição da pesquisa com a proposição de um modelo que integre predição à
API de mapas de posicionamento global, fornecendo informação suficiente a um gestor para decidir quando enviar,
por exemplo, uma frota de caminhões por determinada rodovia que apresente retenções frequentes, ou um usuário da rodovia decidir em qual horário deve viajar. Nesse segundo caso, por exemplo, é comum muitos motoristas optarem por viajar pela manhã bem cedo, quando o sol ainda não está a pique e quando (supostamente) há menos veículos na rua. Nossa pesquisa mostra, por outro lado, que nas primeiras horas da manhã concentra-se um alto número de acidentes, contrariando a preferência que encontramos no senso comum.  

Ainda no que diz respeito aos trabalhos dessa natureza, alguns deles considerados no Estado da Arte dessa dissertação, essa pesquisa avança em relação ao que até então foi proposto em outros trabalhos, pelo fato de que além de identificar ocorrências nas rodovias, levando em conta passado e presente, propõe um modelo que contempla o futuro, possibilitando ao usuário escolhas mais assertivas.

Alguns aspectos merecem destaque em relação às dificuldades enfrentadas na proposição do modelo de classificação e predição. Em primeiro lugar, as informações da base de dados da PRF apresentavam muitas lacunas (missing values), devido ao tipo de registro feito na ocorrência. Por exemplo, um acidente que envolva dois veículos (ou mais) -- um motocicleta e um carro -- aparecia, muitas vezes, com dois registros: acidente envolvendo carro e acidente envolvendo moto. Como essa, várias situações de dados duplicados foram identificados, tornando mais complexo o trabalho de preprocessamento dos dados.

Ainda em relação ao registro dos dados pela PRF, relativos à latitude e longitude estavam ausentes ou mal preenchidos, muitas vezes, de forma equivocada, o que comprometeu a localização exata do evento (colisão, atropelamento), fazendo com que fosse necessário desprezar essas variáveis e encontrar outra alternativa, uma vez que a localização era fundamental para o georreferenciamento.

Outra falta percebida em relação aos dados foi a necessidade de informações quanto ao fluxo de tráfego rodoviário. Para suprir nosso modelo proposto de informações semelhantes e para atingir os objetivos já referidos é que foram construídas as Matrizes de Mortos e Gravidade em cada ponto de cada rodovia, para cada cenário.

Foi percebida também certa imprecisão em relação à quilometragem em que se dava a ocorrência, com erros que variavam de alguns metros até quilômetros, sendo necessário o olhar atento do pesquisador e o confronto cuidadoso de informações, de modo que essas falhas fossem dirimidas.

Outra questão que merece destaque é o complexo processo de limpeza de dados da ``timeline'' do Twitter na etapa de preprocessamento. Atualmente, a própria PRF, para dar mais destaque, tem utilizado imagens para informar as ocorrências nas BRs, em vez de textos, o que tem contribuído para diminuir a quantidade de dados textuais. Com isso foi necessário procurar outras ``timelines'' para ampliar a quantidade de informações do Twitter.

%\pagebreak

\section{Trabalhos futuros}

Esta pesquisa não encerra as questões propostas, relativas ao desenvolvimento de um modelo preditivo. O que foi apresentado, sobretudo, foi a intenção de um modelo que servirá como ponto de partida para o desenvolvimento de uma ferramenta que atenda ao fim proposto que foi propor uma metodologia para classificação e predição de acidentes em rodovias federais brasileiras de forma eficaz. Assim, entendemos que novas pesquisas precisam ser conduzidas para a ampliação do modelo sugerido. 

Trabalhos futuros incluem a incorporação desta proposta em modelos formais de decisão, por exemplo, de roteamento
rodoviário. Através da API do twitter, implementar algoritmos de busca em redes sociais, para encontrar os Hubs difusores de informações e minerar os seus textos, de forma a alimentar a equação do momentum.

Particularmente sobre a equação do momentum, a mesma foi concebida a fim de integrar duas situações espaço-temporais: o passado -- com as bases históricas -- e o futuro -- com as informações das redes sociais -- a ser extrapolado para um painel de controle (dashboard). Esse painel funcionaria da seguinte maneira: se o peso das informações das redes sociais fosse maior do que o peso das informações dos dados históricos, o sistema se guiaria primeiramente pelas redes sociais. Caso contrário, o sistema se guiaria pelos dados históricos. Por exemplo, na ocasião em que o utilizador traça sua rota para uma viagem futura, o Dashboard informa que se guiará pelos dados históricos. No entanto, no momento da viagem, o sistema faz nova varredura nas redes sociais e identifica que haverá uma paralisação no trecho escolhido. Nesse momento ele sugere ao usuário postergar a viagem ou utilizar outra rota. 

O sistema também poderia incorporar a análise sentimental das postagens dos usuários, de maneira que seja possível antecipar atitudes errôneas dos seus utilizadores. Por exemplo, se o utilizador postou que está bebendo em um bar ou que está exausto e com sono, e que vai ``pegar a estrada'' em seguida, e sabendo-se que o itinerário do utilizador passará por pontos já detectados como de alto risco, o sistema emitiria um alerta para o condutor redobrar a atenção ou não conduzir o veículo, nem utilizar a via.

Esse mesmo sistema poderia disponibilizar um conjunto de alertas, quando o condutor traçar um roteiro de viagem, e informar sobre locais de alto risco ou alertar para horários críticos, para que o usuário fique mais atento, reduza a velocidade quando necessário, etc. 

Todas essas sugestões poderiam culminar no desenvolvimento de um aplicativo para Smartphone. O aplicativo disponibilizaria informações antecipadas de bases históricas e de redes sociais, sobre o itinerário escolhido pelo usuário para uma data futura, diferentemente dos aplicativos disponíveis atualmente, em que o usuário consulta sobre rota a ser utilizada no espaço temporal quase imediato. A partir do momento em que o utilizador traçasse a rota para uma viagem a ser realizada, por exemplo dali a uma semana, quinze dias ou outra data futura, escolhendo origem e destino -- semelhante ao que acontece no GPS, Waze, etc. -- o aplicativo automaticamente carregaria as informações de bases históricas em forma de alertas, e ao mesmo tempo faria uma varredura nas redes sociais, para identificar a existência de alguma informação nova, não contemplada na base de dados. 

Outra implementação do aplicativo poderia ser a emissão de relatórios gerencias acerca das viagens anteriores do usuário. Por exemplo, indicar no relatório se as informações disponíveis foram bem utilizadas, ou seja, se o condutor fez uso racional das informações, viajando no melhor horário e evitando trechos perigosos em momentos críticos.

Tudo o que foi discutido anteriormente sugere o desenvolvimento de um sistema de suporte à decisão, que dê apoio ao utilizador para a escolha mais segura, eficaz e econômica. 

Por fim, no trabalho do pesquisador não cabe um ponto final. Quando uma questão de pesquisa é resolvida, novas questões surgem; questões essas que não teriam sido possíveis se não tivesse sido conduzida a pesquisa proposta. Isso reafirma o quanto espírito do pesquisador deve ser o daquele que faz boas perguntas, não apenas o que o que encontra boas respostas. 




